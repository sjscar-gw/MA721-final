\documentclass[11pt]{article}

% Encoding and fonts
\usepackage[T1]{fontenc}
\usepackage[utf8]{inputenc}
\usepackage{lmodern}

\usepackage[margin=1in]{geometry}

\usepackage{amsmath, amssymb, amsthm}
\usepackage{mathtools}

\usepackage{mathrsfs} 
\usepackage{bm}   

\usepackage{enumitem}

\usepackage{graphicx}

\usepackage[colorlinks=true, linkcolor=blue, citecolor=blue]{hyperref}

\newcommand{\R}{\mathbb{R}}
\newcommand{\dd}{\,\mathrm{d}}
\newcommand{\abs}[1]{\left\lvert #1 \right\rvert}
\newcommand{\norm}[1]{\left\lVert #1 \right\rVert}
\newcommand{\ip}[2]{\left\langle #1, #2 \right\rangle}
\newcommand{\ipg}[2]{\left\langle #1, #2 \right\rangle_g}

\numberwithin{equation}{section}
\newtheorem{theorem}{Theorem}[section]
\newtheorem{lemma}[theorem]{Lemma}
\newtheorem{proposition}[theorem]{Proposition}
\newtheorem{corollary}[theorem]{Corollary}

\theoremstyle{definition}
\newtheorem{definition}[theorem]{Definition}
\newtheorem{example}[theorem]{Example}

\theoremstyle{remark}
\newtheorem{remark}[theorem]{Remark}

\title{Geodesics on a Riemannian Manifold via the Calculus of Variations}
\author{Sterling Scarlett Grant Talbert  Arystan Shokan }
\date{December 2025}

\begin{document}
	\maketitle
	
	\tableofcontents
	\pagebreak
	\section{Motivation}
	
	One of the first geometric facts we learn is that, in the Euclidean plane,
	the shortest path between two points is a straight line.  This statement is
	so familiar that it is easy to forget how much structure is hidden in it: we
	are using both the linear structure of $\R^n$ and its standard inner product
	to talk about lengths, angles, and straightness.  As soon as we leave the flat
	world of Euclidean space, the situation becomes less obvious.
	
	For example, on the surface of the Earth, airplanes do not follow straight line
	segments in $\R^3$, but rather arcs of great circles on the sphere.  These
	arcs are locally distance minimizing: near any point on such a curve, if you
	look only at sufficiently short subsegments, they realize the shortest path
	between their endpoints along the surface of the Earth.  From the intrinsic
	point of view of the sphere, they play exactly the role that straight lines
	play in $\R^n$.  This leads to the central question of this paper:
	
	\begin{center}
		\emph{How can we make sense of ``straightest'' or ``shortest'' curves at constant veloctiy on an
			arbitrary smooth manifold?}
	\end{center}
	
	The modern answer begins with the notion of a \emph{Riemannian metric}.
	Informally, a Riemannian metric $g$ on a smooth manifold $M$ gives each
	tangent space $T_pM$ the structure of an inner product space, in a way that
	varies smoothly with the point $p$.  Once such a metric is chosen, we can
	measure the length of tangent vectors, and by integration we obtain lengths of
	curves.  From the lengths of curves we define a distance function $d_g(p,q)$
	by taking an infimum over all curves from $p$ to $q$.
	
	Curves that ``realize'' this distance (or at least make it stationary) are
	called \emph{geodesics}.  Geodesics play the role of straight lines in this
	more general setting.  They are important both for purely geometric reasons
	and for applications: in physics, for instance, geodesics in a Lorentzian
	manifold represent the trajectories of freely-falling particles in general
	relativity.
	
	There are several different but equivalent ways to characterize geodesics:
	
	\begin{itemize}[leftmargin=*]
		\item As locally length-minimizing curves.
		\item As curves with zero covariant acceleration,
		$\nabla_{\dot\gamma}\dot\gamma = 0$.
		\item As critical points of the energy functional
		\[
		E(\gamma)
		= \frac12\int_a^b \norm{\dot\gamma(t)}_g^2 \dd t
		\]
		under variations that fix the endpoints.
	\end{itemize}
	
	The last point is the variational point of view, and it is the main focus of
	this paper.  Instead of guessing the geodesic equation and then justifying it,
	we will start from the length and energy functionals on the space of curves
	and derive the geodesic equation as an Euler--Lagrange equation.  This approach
	fits naturally with classical problems in the calculus of variations and gives
	a conceptually clean derivation of the geodesic equation.
	
	\subsection{A motivating example: the punctured plane}
	
	Before getting into definitions, it is useful to see that the distance function
	behaves in a slightly subtle way even in a very simple example.
	
	\begin{example}[The punctured plane]\label{ex:punctured-plane}
		Let $M = \R^2 \setminus \{0\}$ with the Riemannian metric $g$ induced by
		the standard Euclidean inner product on $\R^2$.  Fix a point
		$p \in M$ and let $q = -p$.  In the full plane $\R^2$, the unique straight
		line segment from $p$ to $q$ has length $2\norm{p}$ and realizes the Euclidean
		distance between $p$ and $q$.
		
		However, in $M$ this straight segment is not allowed, because it passes through
		the origin, which has been removed.  Any piecewise smooth curve
		$\gamma \colon [a,b] \to M$ with $\gamma(a) = p$ and $\gamma(b) = q$ must
		``go around'' the origin.  Intuitively, we still expect the distance between
		$p$ and $q$ to be $2\norm{p}$, but there will be no curve in $M$ that actually
		achieves this length.
	\end{example}
	
	We now make this more precise in one concrete case.
	
	\begin{example}\label{ex:punctured-plane-computation}
		Take $p=(1,0)$ and $q=(-1,0)$, and let $M = \R^2 \setminus \{0\}$ as above.
		For each $\varepsilon > 0$ consider the curve $\gamma_\varepsilon$ that goes
		from $(1,0)$ to $(\varepsilon,0)$ along the $x$-axis, then follows a
		semicircle of radius $\varepsilon$ around the origin to $(-\varepsilon,0)$,
		and then goes from $(-\varepsilon,0)$ to $(-1,0)$ along the $x$-axis.
		
		The first and last segments have total length $2(1-\varepsilon)$, and the
		semicircular arc has length $\pi \varepsilon$.  Therefore
		\[
		L_g(\gamma_\varepsilon)
		= 2(1-\varepsilon) + \pi\varepsilon
		= 2 + (\pi - 2)\varepsilon.
		\]
		As $\varepsilon \to 0^+$ we get $L_g(\gamma_\varepsilon)\to 2$.
		
		On the other hand, every curve in $M$ from $p$ to $q$ must go around the
		origin, and hence has length strictly greater than $2$: if a curve could
		achieve length exactly $2$, it would have to coincide with the straight line
		segment from $(1,0)$ to $(-1,0)$, and that passes through the origin.  Thus
		\[
		d_g(p,q) = 2
		\]
		but there is no curve $\gamma$ in $M$ with $L_g(\gamma) = d_g(p,q)$.  The
		distance is an \emph{infimum} of lengths, not necessarily a \emph{minimum}.
	\end{example}
	
	This example illustrates both the usefulness and the subtlety of the distance
	function.  It also hints at the importance of global assumptions like
	completeness in the study of geodesics.  In complete Riemannian manifolds,
	geodesics are known to realize distances between nearby points, but in
	incomplete manifolds strange things, like the punctured plane, can happen.
	
	\subsection{Goals and outline}
	
	The main goal of this paper is to give a self-contained derivation of the
	geodesic equation on a Riemannian manifold using the calculus of variations,
	and to connect this equation with the intuitive idea of geodesics as shortest
	curves.  Roughly speaking, we will:
	
	\begin{itemize}[leftmargin=*]
		\item Define Riemannian metrics, curve length, and the induced distance
		function $d_g$.
		\item Introduce variations of curves and define the energy functional.
		\item Derive the first variation formula for the energy.
		\item Show that critical points of the energy are precisely smooth curves
		satisfying the geodesic equation $\nabla_{\dot\gamma}\dot\gamma=0$.
		\item Relate geodesics to length-minimizing curves and discuss examples.
	\end{itemize}
	
	The structure of the paper is as follows.  In Section~\ref{sec:prelim} we
	review Riemannian metrics, the length of curves, and the Riemannian distance.
	In Section~\ref{sec:variations} we define variations of curves and the energy
	functional.  Section~\ref{sec:first-variation} is devoted to the first
	variation formula.  In Section~\ref{sec:geodesic-equation} we characterize
	geodesics as curves satisfying a second-order ODE, the geodesic equation, and
	show the equivalence with being critical points of the energy.  In
	Section~\ref{sec:length-vs-energy} we relate energy and length and explain how
	geodesics arise as locally minimizing curves.  Finally, in
	Section~\ref{sec:examples} we compute geodesics explicitly in some important
	examples.
	
	\section{Riemannian metrics, length, and distance}
	\label{sec:prelim}
	
	Throughout, $M$ will denote a smooth ($C^\infty$) manifold of dimension $n$.
	
	\subsection{Riemannian metrics}
	
	\begin{definition}
		A \emph{Riemannian metric} on $M$ is a smooth assignment to each point
		$p\in M$ of an inner product
		\[
		g_p \colon T_pM \times T_pM \to \R
		\]
		such that the map
		\[
		M \times TM\times TM \to \R,\qquad
		(p,v,w) \mapsto g_p(v,w)
		\]
		is smooth when restricted to $TM\times TM$ over $M$.  We usually write
		$\ipg{v}{w} = g_p(v,w)$ when $v,w\in T_pM$, and
		$\norm{v}_g = \sqrt{\ipg{v}{v}}$.
	\end{definition}
	
	In local coordinates, a Riemannian metric is described by a positive-definite
	matrix of smooth functions.
	
	\begin{example}
		Let $(U,(x^1,\dots,x^n))$ be a coordinate chart on $M$.  The coordinate vector
		fields $\partial/\partial x^1,\dots,\partial/\partial x^n$ form a basis for
		$T_pM$ at each $p\in U$.  The metric $g$ is then determined by its components
		\[
		g_{ij}(p) = g_p\!\left(
		\frac{\partial}{\partial x^i}\Big\vert_p,
		\frac{\partial}{\partial x^j}\Big\vert_p
		\right),
		\]
		which form a smooth positive-definite symmetric matrix $(g_{ij})$.  We often
		write
		\[
		g = \sum_{i,j=1}^n g_{ij}(x) \dd x^i \otimes \dd x^j.
		\]
	\end{example}
	
	\subsection{Length of curves}
	
	\begin{definition}
		Let $\gamma\colon [a,b]\to M$ be a piecewise $C^\infty$ curve.  The
		\emph{velocity} of $\gamma$ at $t$ is $\dot\gamma(t) = \dd\gamma/\dd t \in
		T_{\gamma(t)}M$.  The \emph{length} of $\gamma$ with respect to $g$ is
		\[
		L_g(\gamma)
		= \int_a^b \norm{\dot\gamma(t)}_g \dd t
		= \int_a^b \sqrt{
			g_{\gamma(t)}\bigl(\dot\gamma(t),\dot\gamma(t)\bigr)
		}\,\dd t.
		\]
	\end{definition}
	
	It is straightforward to check that the length is invariant under
	orientation-preserving reparametrizations: if $\phi\colon [c,d]\to[a,b]$ is a
	smooth, strictly increasing bijection and $\tilde\gamma = \gamma\circ\phi$,
	then $L_g(\tilde\gamma)=L_g(\gamma)$.
	
	\begin{example}
		In $\R^n$ with the standard metric
		$g = \sum_{i=1}^n \dd x^i\otimes\dd x^i$, the length reduces to the usual
		formula
		\[
		L_g(\gamma)
		= \int_a^b \sqrt{ \sum_{i=1}^n
			\left(\frac{\dd \gamma^i}{\dd t}(t)\right)^2 }\,\dd t,
		\]
		where $\gamma(t) = (\gamma^1(t),\dots,\gamma^n(t))$.
	\end{example}
	
	\subsection{The Riemannian distance function}
	
	\begin{definition}
		Let $(M,g)$ be a connected Riemannian manifold.  For $p,q\in M$ the
		\emph{Riemannian distance} between $p$ and $q$ is
		\[
		d_g(p,q)
		= \inf\{ L_g(\gamma) \mid \gamma\colon [a,b]\to M
		\text{ piecewise $C^\infty$},\ \gamma(a)=p,\ \gamma(b)=q\}.
		\]
	\end{definition}
	
	We now prove that this defines a metric on $M$.
	
	\begin{proposition}
		The function $d_g\colon M\times M\to\R$ is a metric, i.e.
		\begin{enumerate}[label=(\roman*),leftmargin=*]
			\item $d_g(p,q)\ge 0$ and $d_g(p,q)=0$ if and only if $p=q$,
			\item $d_g(p,q) = d_g(q,p)$, and
			\item $d_g(p,r)\le d_g(p,q)+d_g(q,r)$ for all $p,q,r\in M$.
		\end{enumerate}
	\end{proposition}
	
	\begin{proof}
		(i) By definition, $L_g(\gamma)\ge 0$ for every curve $\gamma$, so
		$d_g(p,q)\ge 0$.  If $p=q$, the constant curve $\gamma(t)\equiv p$ has
		length zero, so $d_g(p,p)=0$.
		
		Conversely, suppose $d_g(p,q)=0$.  Let $(U,(x^1,\dots,x^n))$ be a coordinate
		chart containing $p$, and fix a Euclidean norm $\norm{\cdot}$ on $\R^n$.
		Since $g$ is positive definite and smooth, there exist constants
		$0<c<C<\infty$ such that
		\[
		c\norm{v} \le \norm{v}_g \le C\norm{v}
		\]
		for all $v$ in tangent spaces over a small neighborhood of $p$.  If
		$q\ne p$ is sufficiently close to $p$, then there is a smooth curve
		$\gamma$ in $U$ from $p$ to $q$, and its Euclidean length is bounded below by
		some positive number depending on $\norm{x(q)-x(p)}$.  The inequalities above
		imply $L_g(\gamma)\ge c \,L_{\mathrm{Eucl}}(\gamma) >0$, so
		$d_g(p,q)>0$.  Thus if $d_g(p,q)=0$, we must have $p=q$.
		
		(ii) Symmetry is clear: if $\gamma$ is a curve from $p$ to $q$, then the
		reversed curve $\tilde\gamma(t) = \gamma(a+b-t)$ has the same length and goes
		from $q$ to $p$.  Taking infima gives $d_g(p,q)=d_g(q,p)$.
		
		(iii) For the triangle inequality, fix $p,q,r\in M$ and $\varepsilon>0$.
		Choose piecewise smooth curves $\gamma_1$ from $p$ to $q$ and $\gamma_2$ from
		$q$ to $r$ such that
		\[
		L_g(\gamma_1) \le d_g(p,q) + \varepsilon,\qquad
		L_g(\gamma_2) \le d_g(q,r) + \varepsilon.
		\]
		Define the concatenated curve
		\[
		\gamma(t)
		= \begin{cases}
			\gamma_1(2t), & t\in[0,\tfrac12],\\
			\gamma_2(2t-1), & t\in[\tfrac12,1].
		\end{cases}
		\]
		Then
		\[
		L_g(\gamma) = L_g(\gamma_1)+L_g(\gamma_2)
		\le d_g(p,q) + d_g(q,r) + 2\varepsilon.
		\]
		Taking the infimum over all such $\gamma$ gives
		$d_g(p,r) \le d_g(p,q) + d_g(q,r) + 2\varepsilon$, and since $\varepsilon>0$
		was arbitrary, we obtain the triangle inequality.
	\end{proof}
	
	It is also true that the metric topology induced by $d_g$ agrees with the
	original manifold topology.  A full proof requires some more work, but the
	idea is that in local coordinates, the inequality
	$c\norm{v}\le\norm{v}_g\le C\norm{v}$ implies that the $d_g$-balls and the
	Euclidean balls define the same notion of ``small neighborhood.''  We will not
	need the precise details later.
	
	\section{Variations and the energy functional}
	\label{sec:variations}
	
	In order to find critical points of the length or energy functional, we need a
	way to perturb a given curve.
	
	\subsection{Variations of curves}
	
	\begin{definition}
		Let $\gamma\colon [a,b]\to M$ be a piecewise $C^\infty$ curve.  A
		\emph{variation} of $\gamma$ is a smooth map
		\[
		\alpha\colon (-\varepsilon,\varepsilon)\times[a,b]\to M
		\]
		for some $\varepsilon>0$ such that:
		\begin{enumerate}[label=(\roman*),leftmargin=*]
			\item $\alpha(0,t) = \gamma(t)$ for all $t\in[a,b]$;
			\item There is a partition
			\[
			a = t_0 < t_1 < \dots < t_N = b
			\]
			such that for each fixed $u$, the curve
			\[
			\bar\alpha(u)\colon [a,b]\to M,\qquad
			\bar\alpha(u)(t) = \alpha(u,t),
			\]
			is $C^\infty$ on each closed subinterval $[t_i,t_{i+1}]$.
		\end{enumerate}
		We say that the variation \emph{fixes endpoints} if $\alpha(u,a)$ and
		$\alpha(u,b)$ are independent of $u$.
	\end{definition}
	
	\begin{definition}
		Let $\alpha$ be a variation of $\gamma$.  The \emph{variation vector field}
		along $\gamma$ is the vector field $V$ along $\gamma$ defined by
		\[
		V(t)
		= \left.\frac{\partial\alpha}{\partial u}\right\vert_{u=0}(t)
		\in T_{\gamma(t)}M.
		\]
		If $\alpha$ fixes endpoints, then $V(a)=V(b)=0$.
	\end{definition}
	
	\begin{remark}
		Conversely, given a sufficiently nice vector field $V$ along $\gamma$, one
		can construct a variation $\alpha$ with variation field $V$.  We will use this
		in the proof of the geodesic equation.
	\end{remark}
	
	\subsection{The energy functional}
	
	Directly working with the length functional
	\[
	L_g(\gamma)
	= \int_a^b \norm{\dot\gamma(t)}_g \dd t
	\]
	is possible but somewhat technical because of the square root.  It is more
	convenient to work with the \emph{energy functional}, whose integrand is
	quadratic in the velocity.
	
	\begin{definition}
		Let $\gamma\colon[a,b]\to M$ be piecewise $C^\infty$.  The
		\emph{energy} of $\gamma$ is
		\[
		E_{a}^{b}(\gamma)
		= \frac12 \int_a^b \norm{\dot\gamma(t)}_g^2 \dd t
		= \frac12 \int_a^b
		g_{\gamma(t)}\bigl(\dot\gamma(t),\dot\gamma(t)\bigr)\,\dd t.
		\]
	\end{definition}
	
	If we fix endpoints $p,q\in M$ and an interval $[a,b]$, we can think of
	$E_a^b$ as a functional on the space of piecewise $C^\infty$ curves
	$\gamma\colon[a,b]\to M$ with $\gamma(a)=p$ and $\gamma(b)=q$.  We will say
	that $\gamma$ is a \emph{critical point} of $E_a^b$ if the derivative
	\[
	\left.\frac{\dd}{\dd u} E_a^b(\bar\alpha(u))\right\vert_{u=0}
	\]
	vanishes for every variation $\alpha$ that fixes endpoints.
	
	The energy and length functionals are closely related.
	
	\begin{proposition}\label{prop:length-energy-ineq}
		Let $\gamma\colon[a,b]\to M$ be piecewise $C^\infty$.  Then
		\[
		L_g(\gamma)^2
		\le 2(b-a)\,E_a^b(\gamma),
		\]
		with equality if and only if $\norm{\dot\gamma(t)}_g$ is constant on $[a,b]$.
	\end{proposition}
	
	\begin{proof}
		Set $f(t) = \norm{\dot\gamma(t)}_g$.  By Cauchy--Schwarz,
		\[
		\left( \int_a^b f(t)\,\dd t \right)^2
		\le (b-a)\int_a^b f(t)^2\,\dd t,
		\]
		with equality if and only if $f$ is constant almost everywhere.  The left-hand
		side is $L_g(\gamma)^2$, while the right-hand side is
		\[
		(b-a)\int_a^b f(t)^2\,\dd t
		= (b-a)\int_a^b \norm{\dot\gamma(t)}_g^2\,\dd t
		= 2(b-a)\,E_a^b(\gamma).
		\]
		The condition for equality is exactly that $\norm{\dot\gamma(t)}_g$ be
		constant.
	\end{proof}
	
	This shows that among curves with fixed endpoints and fixed parameter
	interval $[a,b]$, those with constant speed are the ones for which energy and
	length are most tightly related.  In particular, any curve that minimizes
	energy among all curves with the same endpoints and parameter interval must
	have constant speed.
	
	\section{The first variation of energy}
	\label{sec:first-variation}
	
	We now compute the derivative of the energy functional along a variation.
	This is the key step in deriving the geodesic equation.
	
	\subsection{Left and right derivatives}
	
	Because we allow piecewise smooth curves, it is useful to introduce the
	notation for left and right derivatives at the break points.
	
	Let $\gamma\colon[a,b]\to M$ be piecewise $C^\infty$ with partition
	$a=t_0<\dots<t_N=b$ such that $\gamma$ is $C^\infty$ on each open subinterval
	$(t_i,t_{i+1})$.
	
	\begin{definition}
		For $1\le i\le N-1$, the \emph{left} and \emph{right} derivatives of $\gamma$
		at $t_i$ are defined by
		\[
		\dot\gamma(t_i^-)
		= \lim_{t\to t_i^-} \dot\gamma(t),\qquad
		\dot\gamma(t_i^+)
		= \lim_{t\to t_i^+} \dot\gamma(t),
		\]
		computed in local coordinates.  We define the \emph{jump} at $t_i$ by
		\[
		\Delta_{t_i}\dot\gamma
		= \dot\gamma(t_i^+) - \dot\gamma(t_i^-).
		\]
		We also set
		\[
		\Delta_{t_0}\dot\gamma = \dot\gamma(t_0^+),\qquad
		\Delta_{t_N}\dot\gamma = -\dot\gamma(t_N^-).
		\]
	\end{definition}
	
	If $\gamma$ is $C^\infty$ on the entire interval $[a,b]$, then all jumps
	$\Delta_{t_i}\dot\gamma$ vanish.
	
	\subsection{A key lemma from the calculus of variations}
	
	We will repeatedly use the following linear-algebraic lemma.
	
	\begin{lemma}\label{lem:test-fields}
		Let $W\colon[a,b]\to\R^n$ be continuous.  Suppose
		\[
		\int_a^b \ip{V(t)}{W(t)} \dd t = 0
		\]
		for every smooth $V\colon[a,b]\to\R^n$ with $V(a)=V(b)=0$.  Then
		$W(t)\equiv 0$ on $[a,b]$.
	\end{lemma}
	
	\begin{proof}
		Fix $t_0\in(a,b)$ and suppose $W(t_0)\ne 0$.  Let $e_1,\dots,e_n$ be the
		standard basis of $\R^n$, and write $W(t_0) = \sum_{k=1}^n w_k e_k$.  Then
		at least one $w_k$ is nonzero; without loss of generality $w_1\ne 0$.
		
		By continuity, there exists $\delta>0$ such that
		$\ip{e_1}{W(t)}$ has the same sign as $w_1$ for all
		$t\in(t_0-\delta,t_0+\delta)$.  Choose a smooth bump function
		$\varphi\colon[a,b]\to\R$ such that $\varphi$ is supported in
		$(t_0-\delta,t_0+\delta)$ and $\varphi(t)\ge 0$ with $\varphi(t_0)>0$.
		
		Define $V(t) = \varphi(t)e_1$.  Then $V$ is smooth, $V(a)=V(b)=0$, and
		\[
		\int_a^b \ip{V(t)}{W(t)} \dd t
		= \int_a^b \varphi(t)\ip{e_1}{W(t)} \dd t.
		\]
		The integrand is nonnegative and positive near $t_0$, so the integral is
		strictly positive.  This contradicts the assumption that the integral is
		always zero.  Hence $W(t)=0$ for all $t$.
	\end{proof}
	
	The same argument works for vector fields along a curve in a manifold if we
	use a coordinate chart.
	
	\subsection{The first variation formula}
	
	We now state and prove the first variation formula.  The proof involves a
	coordinate computation on each subinterval and an integration by parts.
	
	\begin{theorem}[First variation of energy]\label{thm:first-variation}
		Let $(M,g)$ be a Riemannian manifold and
		$\gamma\colon[a,b]\to M$ a piecewise $C^\infty$ curve with partition
		$a=t_0<\dots<t_N=b$.  Let $\alpha$ be a variation of $\gamma$ with variation
		vector field $V$.  Then
		\begin{align*}
			\left.\frac{\dd}{\dd u} E_a^b(\bar\alpha(u))\right\vert_{u=0}
			&= -\int_a^b \ipg{V(t)}{\nabla_{\dot\gamma}\dot\gamma(t)} \dd t \\
			&\quad - \sum_{i=0}^N
			\ipg{V(t_i)}{\Delta_{t_i}\dot\gamma}.
		\end{align*}
		Here $\nabla$ is the Levi-Civita connection of $g$.
	\end{theorem}
	
	\begin{proof}
		We first express the energy in local coordinates and then differentiate.
		
		Choose a coordinate chart $(U,(x^1,\dots,x^n))$ such that
		$\gamma([t_i,t_{i+1}])\subset U$.  We may refine the partition if necessary
		to ensure this.  Write
		\[
		\gamma(t) = \bigl(\gamma^1(t),\dots,\gamma^n(t)\bigr)
		\]
		and
		\[
		\dot\gamma(t)
		= \sum_{k=1}^n \frac{\dd\gamma^k}{\dd t}(t)
		\frac{\partial}{\partial x^k}\Big\vert_{\gamma(t)}.
		\]
		Let $g_{ij}$ be the components of $g$ in these coordinates.  Then the
		integrand of the energy on $[t_i,t_{i+1}]$ is
		\[
		\frac12\norm{\dot\gamma(t)}_g^2
		= \frac12\sum_{j,\ell=1}^n
		g_{j\ell}(\gamma(t))
		\frac{\dd\gamma^j}{\dd t}(t)
		\frac{\dd\gamma^\ell}{\dd t}(t).
		\]
		Define
		\[
		F(x,y)
		= \frac12 \sum_{j,\ell=1}^n g_{j\ell}(x) y^j y^\ell,
		\qquad x\in U,\ y\in\R^n.
		\]
		Then
		\[
		E_a^b(\gamma)
		= \sum_{i=0}^{N-1}
		\int_{t_i}^{t_{i+1}}
		F\bigl(\gamma(t),\dot\gamma(t)\bigr)\,\dd t.
		\]
		
		Now consider a variation $\alpha(u,t)$ with $\alpha(0,t)=\gamma(t)$.  In
		coordinates, write
		\[
		\alpha(u,t) = \bigl(\alpha^1(u,t),\dots,\alpha^n(u,t)\bigr).
		\]
		For each fixed $u$, we have a curve $t\mapsto\alpha(u,t)$, and we denote its
		components by $\alpha^k(u,t)$.  The velocity is
		\[
		\frac{\partial\alpha}{\partial t}(u,t)
		= \sum_{k=1}^n
		\frac{\partial\alpha^k}{\partial t}(u,t)
		\frac{\partial}{\partial x^k}\Big\vert_{\alpha(u,t)}.
		\]
		
		Then
		\[
		E_a^b(\bar\alpha(u))
		= \sum_{i=0}^{N-1}
		\int_{t_i}^{t_{i+1}}
		F\left(\alpha(u,t),
		\frac{\partial\alpha}{\partial t}(u,t)\right)\,\dd t.
		\]
		Differentiating with respect to $u$ under the integral sign, we get
		\[
		\frac{\dd}{\dd u} E_a^b(\bar\alpha(u))
		= \sum_{i=0}^{N-1}
		\int_{t_i}^{t_{i+1}}
		\left[
		\sum_{k=1}^n
		\frac{\partial F}{\partial x^k}
		\frac{\partial\alpha^k}{\partial u}
		+ \sum_{k=1}^n
		\frac{\partial F}{\partial y^k}
		\frac{\partial}{\partial t}
		\left(\frac{\partial\alpha^k}{\partial u}\right)
		\right]\dd t.
		\]
		Evaluating at $u=0$ and using that $\alpha(0,t)=\gamma(t)$, we find
		\begin{align*}
			\left.\frac{\dd}{\dd u} E_a^b(\bar\alpha(u))\right\vert_{u=0}
			&= \sum_{i=0}^{N-1}
			\int_{t_i}^{t_{i+1}}
			\left[
			\sum_{k=1}^n
			\left.\frac{\partial F}{\partial x^k}\right\vert_{(\gamma,\dot\gamma)}
			\frac{\partial\alpha^k}{\partial u}(0,t)
			\right.\\
			&\qquad\qquad\left.
			+ \sum_{k=1}^n
			\left.\frac{\partial F}{\partial y^k}\right\vert_{(\gamma,\dot\gamma)}
			\frac{\partial}{\partial t}
			\left(\frac{\partial\alpha^k}{\partial u}(0,t)\right)
			\right]\dd t.
		\end{align*}
		We integrate the second term by parts on each interval
		$[t_i,t_{i+1}]$:
		\begin{align*}
			\int_{t_i}^{t_{i+1}}
			\sum_{k=1}^n
			\frac{\partial F}{\partial y^k}
			\frac{\partial}{\partial t}
			\left(\frac{\partial\alpha^k}{\partial u}(0,t)\right)\,\dd t
			&= \left[
			\sum_{k=1}^n
			\frac{\partial F}{\partial y^k}
			\frac{\partial\alpha^k}{\partial u}(0,t)
			\right]_{t_i}^{t_{i+1}} \\
			&\quad - \int_{t_i}^{t_{i+1}}
			\sum_{k=1}^n
			\frac{\partial}{\partial t}
			\left(\frac{\partial F}{\partial y^k}\right)
			\frac{\partial\alpha^k}{\partial u}(0,t)\,\dd t.
		\end{align*}
		
		Summing over $i$ and combining terms, we obtain
		\begin{align*}
			\left.\frac{\dd}{\dd u} E_a^b(\bar\alpha(u))\right\vert_{u=0}
			&= \sum_{i=0}^{N-1}
			\int_{t_i}^{t_{i+1}}
			\sum_{k=1}^n
			\left(
			\frac{\partial F}{\partial x^k}
			- \frac{\partial}{\partial t}
			\left(\frac{\partial F}{\partial y^k}\right)
			\right)
			\frac{\partial\alpha^k}{\partial u}(0,t)\,\dd t \\
			&\quad + \sum_{i=0}^{N-1}
			\left[
			\sum_{k=1}^n
			\frac{\partial F}{\partial y^k}
			\frac{\partial\alpha^k}{\partial u}(0,t)
			\right]_{t_i}^{t_{i+1}}.
		\end{align*}
		
		The last sum telescopes.  Evaluating the boundary terms at the internal points
		$t_1,\dots,t_{N-1}$ yields expressions involving the left and right
		derivatives of $\gamma$ at those points.  One checks that this boundary
		contribution can be written as
		\[
		-\sum_{i=0}^N
		\ipg{V(t_i)}{\Delta_{t_i}\dot\gamma},
		\]
		where $V$ is the variation vector field and $\Delta_{t_i}\dot\gamma$ is the
		jump in the velocity at $t_i$.  We omit some routine algebra; the main point
		is that the terms at $t_i$ involve differences of the form
		\[
		\ipg{V(t_i)}{\dot\gamma(t_i^+)}-
		\ipg{V(t_i)}{\dot\gamma(t_i^-)}
		= \ipg{V(t_i)}{\Delta_{t_i}\dot\gamma}.
		\]
		
		The remaining integral gives the Euler--Lagrange part.  A direct
		computation shows that
		\[
		\frac{\partial F}{\partial x^k}
		- \frac{\partial}{\partial t}
		\left(\frac{\partial F}{\partial y^k}\right)
		= -\ipg{
			\frac{\partial}{\partial x^k}
		}{
			\nabla_{\dot\gamma}\dot\gamma
		},
		\]
		where $\nabla$ is the Levi-Civita connection.  Therefore
		\[
		\sum_{k=1}^n
		\left(
		\frac{\partial F}{\partial x^k}
		- \frac{\partial}{\partial t}
		\left(\frac{\partial F}{\partial y^k}\right)
		\right)
		\frac{\partial\alpha^k}{\partial u}(0,t)
		= -\ipg{V(t)}{\nabla_{\dot\gamma}\dot\gamma(t)}.
		\]
		Substituting this back into the expression for the derivative completes the
		proof.
	\end{proof}
	
	\section{Geodesics and the geodesic equation}
	\label{sec:geodesic-equation}
	
	We can now define geodesics and characterize them as solutions of a second
	order ODE.
	
	\begin{definition}
		A piecewise $C^\infty$ curve $\gamma\colon[a,b]\to M$ is called a
		\emph{geodesic} if it is a critical point of the energy functional $E_a^b$
		among all variations that fix endpoints.
	\end{definition}
	
	Using the first variation formula, we obtain a more geometric description.
	
	\begin{theorem}\label{thm:geodesic-iff}
		A piecewise $C^\infty$ curve $\gamma\colon[a,b]\to M$ is a critical point of
		$E_a^b$ for fixed endpoints if and only if:
		\begin{enumerate}[label=(\roman*),leftmargin=*]
			\item $\gamma$ is in fact $C^\infty$ on $[a,b]$ (so all jumps
			$\Delta_{t_i}\dot\gamma$ vanish), and
			\item $\gamma$ satisfies the \emph{geodesic equation}
			\[
			\nabla_{\dot\gamma}\dot\gamma(t) = 0
			\quad \text{for all } t\in[a,b].
			\]
		\end{enumerate}
	\end{theorem}
	
	\begin{proof}
		Assume $\gamma$ is a critical point of $E_a^b$.  Let $\alpha$ be any
		variation of $\gamma$ fixing endpoints, with variation vector field $V$.  Then
		by Theorem~\ref{thm:first-variation},
		\[
		0
		= \left.\frac{\dd}{\dd u} E_a^b(\bar\alpha(u))\right\vert_{u=0}
		= -\int_a^b \ipg{V}{\nabla_{\dot\gamma}\dot\gamma} \dd t
		- \sum_{i=0}^N \ipg{V(t_i)}{\Delta_{t_i}\dot\gamma}.
		\]
		
		We first show that the jump terms vanish.  Fix an index $i$ and choose a
		variation $\alpha$ whose variation field $V$ is supported very close to $t_i$,
		with $V(a)=V(b)=0$ and $V(t_j)=0$ for all $j\ne i$.  (This can be done by
		building $V$ from a bump function and then integrating to get $\alpha$; see
		the remark below.)  For such a variation, the integral term can be made
		arbitrarily small, but the sum over $i$ reduces to the single term
		$-\ipg{V(t_i)}{\Delta_{t_i}\dot\gamma}$.  Since the whole expression must be
		zero for all such $V$, we conclude that
		\[
		\ipg{V(t_i)}{\Delta_{t_i}\dot\gamma} = 0
		\]
		for all $V(t_i)\in T_{\gamma(t_i)}M$.  This forces $\Delta_{t_i}\dot\gamma=0$.
		
		Thus $\dot\gamma(t)$ has no jumps and is continuous on $[a,b]$.  Since
		$\gamma$ is $C^\infty$ on each subinterval and $\dot\gamma$ is continuous at
		the break points, standard results on ODEs imply that $\gamma$ is in fact
		$C^\infty$ on all of $[a,b]$.
		
		With $\gamma$ now smooth, all $\Delta_{t_i}\dot\gamma=0$, so the first
		variation formula simplifies to
		\[
		\int_a^b \ipg{V}{\nabla_{\dot\gamma}\dot\gamma} \dd t = 0
		\]
		for every smooth vector field $V$ along $\gamma$ with $V(a)=V(b)=0$.  By
		working in local coordinates and applying Lemma~\ref{lem:test-fields}, we see
		that this forces $\nabla_{\dot\gamma}\dot\gamma(t)=0$ for all $t$.
		
		Conversely, suppose $\gamma$ is $C^\infty$ and satisfies
		$\nabla_{\dot\gamma}\dot\gamma=0$.  Then all jumps vanish, and the integral
		term in the first variation formula also vanishes for any variation $\alpha$
		(with fixed endpoints).  Hence $\dd E_a^b(\bar\alpha(u))/\dd u|_{u=0}=0$ for
		all $\alpha$, so $\gamma$ is a critical point.
	\end{proof}
	
	\begin{remark}
		To justify the existence of variations with a prescribed variation field $V$,
		one can proceed as follows.  In local coordinates, define
		\[
		\alpha(u,t)
		= \exp_{\gamma(t)}(uV(t)),
		\]
		where $\exp$ is the exponential map associated to $g$.  For small $u$, this is
		well-defined and yields a smooth map $\alpha$ with $\alpha(0,t)=\gamma(t)$ and
		variation field $V$.  If we want the endpoints fixed, we arrange $V(a)=V(b)=0$.
		We will not develop the full theory of the exponential map here; instead, we
		can work locally and patch together variations using bump functions.
	\end{remark}
	
	In local coordinates, the geodesic equation takes a more explicit form.
	
	\begin{proposition}[Geodesic equation in coordinates]
		Let $(x^1,\dots,x^n)$ be local coordinates on $M$, and let $g_{ij}$ be the
		components of the metric in these coordinates.  Let $(g^{ij})$ be the inverse
		matrix.  The Christoffel symbols of the Levi-Civita connection are given by
		\[
		\Gamma^k_{ij}
		= \frac12 \sum_{\ell=1}^n g^{k\ell}
		\left(
		\frac{\partial g_{j\ell}}{\partial x^i}
		+ \frac{\partial g_{i\ell}}{\partial x^j}
		- \frac{\partial g_{ij}}{\partial x^\ell}
		\right).
		\]
		A smooth curve $\gamma\colon[a,b]\to M$ with coordinate representation
		$\gamma(t) = (\gamma^1(t),\dots,\gamma^n(t))$ is a geodesic if and only if its
		components satisfy
		\[
		\frac{\dd^2\gamma^k}{\dd t^2}(t)
		+ \sum_{i,j=1}^n
		\Gamma^k_{ij}\bigl(\gamma(t)\bigr)
		\frac{\dd\gamma^i}{\dd t}(t)
		\frac{\dd\gamma^j}{\dd t}(t)
		= 0,
		\qquad k=1,\dots,n.
		\]
	\end{proposition}
	
	\begin{proof}
		In local coordinates, the covariant derivative of $\dot\gamma$ along itself is
		given by
		\[
		\nabla_{\dot\gamma}\dot\gamma
		= \sum_{k=1}^n
		\left(
		\frac{\dd^2\gamma^k}{\dd t^2}
		+ \sum_{i,j=1}^n
		\Gamma^k_{ij}(\gamma(t))
		\frac{\dd\gamma^i}{\dd t}
		\frac{\dd\gamma^j}{\dd t}
		\right)
		\frac{\partial}{\partial x^k}\Big\vert_{\gamma(t)}.
		\]
		Thus $\nabla_{\dot\gamma}\dot\gamma=0$ if and only if each coordinate
		component satisfies the claimed ODE.
	\end{proof}
	
	\section{Length versus energy and minimizing properties}
	\label{sec:length-vs-energy}
	
	We now relate the variational characterization of geodesics (via energy) to
	the more geometric intuition of geodesics as locally length-minimizing curves.
	
	\subsection{Arc-length parametrization}
	
	Let $\gamma\colon[a,b]\to M$ be a smooth curve with $\dot\gamma(t)\ne 0$ for
	all $t$.  Define its \emph{arc-length parameter} by
	\[
	s(t) = \int_a^t \norm{\dot\gamma(\tau)}_g \dd\tau.
	\]
	Then $s\colon[a,b]\to[0,L_g(\gamma)]$ is smooth and strictly increasing, hence
	a diffeomorphism onto its image.  We can invert it to obtain $t(s)$ and define
	the reparametrized curve
	\[
	\tilde\gamma(s) = \gamma(t(s)).
	\]
	By construction, $\tilde\gamma$ has unit speed:
	\[
	\norm{\dot{\tilde\gamma}(s)}_g
	= 1
	\]
	for all $s$.  In particular, $L_g(\tilde\gamma)=L_g(\gamma)$.
	
	\subsection{Critical points of length vs.\ critical points of energy}
	
	The length functional is less convenient to differentiate because of the
	square root, but for unit-speed curves it is closely related to the energy.
	
	\begin{theorem}\label{thm:length-critical}
		Let $\gamma\colon[a,b]\to M$ be a smooth curve with $\dot\gamma(t)\ne 0$ for
		all $t$, and let $\tilde\gamma$ be its arc-length reparametrization.  Then:
		\begin{enumerate}[label=(\roman*),leftmargin=*]
			\item $\gamma$ is a critical point of the energy functional $E_a^b$ with
			fixed endpoints if and only if its arc-length reparametrization
			$\tilde\gamma$ is a critical point of the length functional.
			\item Any smooth curve $\sigma$ that is critical for the length functional
			(with fixed endpoints) and has nonvanishing velocity can be
			reparametrized to a geodesic.
		\end{enumerate}
	\end{theorem}
	
	\begin{proof}[Sketch of proof]
		A full proof requires computing the first variation of the length functional,
		which is similar to, but slightly more complicated than, the computation for
		the energy.  The idea is as follows.
		
		For a unit-speed curve $\tilde\gamma$, the length and energy functionals
		satisfy
		\[
		L_g(\tilde\gamma)
		= \int_0^{L_g(\gamma)} 1\,\dd s
		= L_g(\gamma),
		\]
		and
		\[
		E_0^{L_g(\gamma)}(\tilde\gamma)
		= \frac12\int_0^{L_g(\gamma)} 1^2\,\dd s
		= \frac12 L_g(\gamma).
		\]
		Thus up to a constant factor, $L$ and $E$ coincide on unit-speed curves.
		The first variation of $L$ along variations that respect unit speed is
		proportional to the first variation of $E$.
		
		More concretely, if $\tilde\alpha$ is a variation of $\tilde\gamma$ through
		unit-speed curves with fixed endpoints, then
		\[
		\left.\frac{\dd}{\dd u} L_g(\tilde\alpha(u))\right\vert_{u=0}
		= \frac{1}{\norm{\dot{\tilde\gamma}}^2_g}
		\left.\frac{\dd}{\dd u}
		E(\tilde\alpha(u))\right\vert_{u=0}.
		\]
		Therefore, the vanishing of the first variation of $E$ is equivalent to the
		vanishing of the first variation of $L$ under such variations.
		
		The second statement follows by applying arc-length reparametrization and then
		using the characterization of geodesics as critical points of $E$ from
		Theorem~\ref{thm:geodesic-iff}.  Details can be filled in by adapting the
		proof of the first variation formula to $L$ instead of $E$.
	\end{proof}
	
	\subsection{Local minimizing property of geodesics}
	
	In general, geodesics are critical points of the length functional, not
	necessarily global minimizers.  Nevertheless, sufficiently short geodesic
	segments do minimize length between their endpoints.
	
	\begin{theorem}[Local minimizing property]
		Let $(M,g)$ be a Riemannian manifold and $p\in M$.  There exists a
		neighborhood $U$ of $p$ such that for any $q\in U$ there is a unique
		geodesic segment $\gamma$ from $p$ to $q$ lying in $U$, and $\gamma$ is the
		unique minimizing curve between $p$ and $q$.
	\end{theorem}
	
	\begin{proof}[Idea of proof]
		The proof uses the exponential map $\exp_p\colon T_pM\to M$, which is defined
		by $\exp_p(v)=\gamma_v(1)$, where $\gamma_v$ is the unique geodesic with
		$\gamma_v(0)=p$ and $\dot\gamma_v(0)=v$.  For sufficiently small $v$, the map
		$\exp_p$ is a diffeomorphism onto a neighborhood $U$ of $p$.
		
		Geodesics through $p$ correspond to straight lines in $T_pM$ in these
		coordinates, and one can use this to show that the radial geodesic from $p$ to
		$q$ is the unique minimizing curve in $U$.  The full proof requires some
		results from ODE theory and differential topology and is usually given in a
		course in Riemannian geometry.  We will not reproduce all technical details
		here, but the key point is that geodesics are indeed locally shortest paths.
	\end{proof}
	
	Our focus in this paper is on the variational derivation of the geodesic
	equation rather than on these global existence and uniqueness results, so we
	stop here.
	
	\section{Examples}
	\label{sec:examples}
	
	We now compute geodesics explicitly in some important examples, illustrating
	how the general theory plays out in practice.
	
	\subsection{Euclidean space \texorpdfstring{$\R^n$}{Rn}}
	
	Consider $M=\R^n$ with the standard Euclidean metric
	\[
	g = \sum_{i=1}^n \dd x^i\otimes\dd x^i.
	\]
	In the standard coordinates $(x^1,\dots,x^n)$ we have $g_{ij}=\delta_{ij}$.
	All partial derivatives $\partial g_{ij}/\partial x^k$ are zero, so all
	Christoffel symbols vanish:
	\[
	\Gamma^k_{ij} = 0 \quad\text{for all } i,j,k.
	\]
	
	The geodesic equation therefore reduces to
	\[
	\frac{\dd^2\gamma^k}{\dd t^2}
	= 0,\qquad k=1,\dots,n.
	\]
	The general solution is
	\[
	\gamma^k(t) = A^k t + B^k,
	\]
	so $\gamma$ is an affine map:
	\[
	\gamma(t) = At + B
	\]
	for some constant vectors $A,B\in\R^n$.  These are precisely straight lines.
	Thus our abstract definition of geodesics recovers the familiar fact that
	geodesics in Euclidean space are straight lines with constant velocity.
	
	\subsection{The round two-sphere \texorpdfstring{$S^2$}{S2}}
	
	Let $M=S^2\subset\R^3$ be the unit sphere with the metric induced by the
	Euclidean inner product.  To compute the geodesic equation, we introduce
	spherical coordinates.
	
	Write a point on $S^2$ as
	\[
	(\cos\theta\cos\varphi,\ \cos\theta\sin\varphi,\ \sin\theta),
	\]
	where $\theta\in(-\tfrac{\pi}{2},\tfrac{\pi}{2})$ is latitude and
	$\varphi\in(0,2\pi)$ is longitude.  In these coordinates, a straightforward
	computation shows that the induced metric is
	\[
	g = \dd\theta^2 + \cos^2\theta\,\dd\varphi^2.
	\]
	(If one uses colatitude instead of latitude, one gets
	$g = \dd\theta^2 + \sin^2\theta\,\dd\varphi^2$; both descriptions are
	equivalent up to a coordinate change.)
	
	For definiteness, let us take
	\[
	g_{\theta\theta}=1,\qquad
	g_{\varphi\varphi}=\sin^2\theta,\qquad
	g_{\theta\varphi}=g_{\varphi\theta}=0.
	\]
	Then the inverse matrix has
	\[
	g^{\theta\theta}=1,\qquad
	g^{\varphi\varphi}=\frac{1}{\sin^2\theta},\qquad
	g^{\theta\varphi}=g^{\varphi\theta}=0.
	\]
	
	Using the formula for Christoffel symbols, we compute the nonzero ones:
	\begin{align*}
		\Gamma^{\theta}_{\varphi\varphi}
		&= \frac12 g^{\theta\theta}
		\left(
		\frac{\partial g_{\varphi\theta}}{\partial x^\varphi}
		+ \frac{\partial g_{\varphi\theta}}{\partial x^\varphi}
		- \frac{\partial g_{\varphi\varphi}}{\partial x^\theta}
		\right)
		= -\sin\theta\cos\theta,\\[0.5em]
		\Gamma^{\varphi}_{\theta\varphi}
		&= \Gamma^{\varphi}_{\varphi\theta}
		= \frac12 g^{\varphi\varphi}
		\left(
		\frac{\partial g_{\varphi\varphi}}{\partial x^\theta}
		+ \frac{\partial g_{\theta\varphi}}{\partial x^\varphi}
		- \frac{\partial g_{\theta\varphi}}{\partial x^\varphi}
		\right)
		= \cot\theta.
	\end{align*}
	All other $\Gamma^k_{ij}$ vanish.
	
	Therefore, a curve $\gamma(t)=(\theta(t),\varphi(t))$ on $S^2$ is a geodesic if
	and only if it satisfies
	\begin{align*}
		\frac{\dd^2\theta}{\dd t^2}
		- \sin\theta\cos\theta
		\left(\frac{\dd\varphi}{\dd t}\right)^2 &= 0,\\[0.5em]
		\frac{\dd^2\varphi}{\dd t^2}
		+ 2\cot\theta\,\frac{\dd\theta}{\dd t}\,\frac{\dd\varphi}{\dd t} &= 0.
	\end{align*}
	
	Solving this system explicitly is somewhat involved, but one can show that its
	solutions are exactly the great circles on $S^2$.  For instance, if we fix a
	unit vector $u\in\R^3$, then the intersection of $S^2$ with the plane through
	the origin orthogonal to $u$ is a great circle.  Parameterizing this curve
	with constant speed yields a solution of the geodesic equation.  Conversely,
	one can prove that any geodesic on the round sphere extends uniquely to a
	great circle.
	
	\subsection{The punctured plane revisited}
	
	We return to the punctured plane
	\[
	M = \R^2 \setminus\{0\}
	\]
	with the induced Euclidean metric.  In local coordinates away from the origin,
	the metric is just the standard Euclidean one, so geodesics are locally
	straight lines.  However, global properties of these geodesics can be more
	complicated.
	
	If $p$ and $q$ lie on the same ray emanating from the origin (say $q=\lambda
	p$ for some $\lambda>0$), then the straight segment from $p$ to $q$ does not
	hit the origin and is a geodesic in $M$ that minimizes length between its
	endpoints.  On the other hand, if $q=-p$ as in
	Example~\ref{ex:punctured-plane-computation}, then any straight segment from
	$p$ to $q$ passes through $0$ and is not contained in $M$.  There are many
	geodesics passing near the origin, but none of them connect $p$ to $q$ in a
	way that realizes the distance $d_g(p,q)$.
	
	This illustrates that the existence of minimizing geodesics between arbitrary
	pairs of points depends on global conditions like completeness.  The punctured
	plane is incomplete, and the failure of geodesics to realize distances between
	some points is one manifestation of this incompleteness.
	
	\section*{Conclusion}
	
	In this paper we have developed the variational approach to geodesics on
	Riemannian manifolds.  Starting from the definition of a Riemannian metric, we
	introduced the length and energy functionals on the space of curves, computed
	the first variation of the energy, and derived the geodesic equation as the
	Euler--Lagrange equation for critical points of the energy with fixed
	endpoints.  We saw that geodesics can be characterized intrinsically as curves
	with zero covariant acceleration, and that in local coordinates they satisfy a
	second-order system of ordinary differential equations involving the
	Christoffel symbols.
	
	We also discussed the relationship between energy and length, showing that
	constant-speed geodesics are critical points of the length functional and are
	locally minimizing curves.  Finally, we computed geodesics in several
	important examples, including Euclidean space, the round sphere, and the
	punctured plane, illustrating how both local and global geometric features
	affect the behavior of geodesics.
	
	Many further directions are possible: the study of Jacobi fields and
	conjugate points, Morse theory on loop spaces, comparison theorems relating
	curvature to the behavior of geodesics, and the role of geodesics in physical
	theories such as general relativity.  The variational methods and basic
	geometric ideas developed here provide a foundation for exploring these more
	advanced topics.
	
\end{document}
