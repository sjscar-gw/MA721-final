	\section{Examples}
	\label{sec:examples}
	
	We now compute geodesics explicitly in some important examples, illustrating
	how the general theory plays out in practice.
	
	\subsection{Euclidean space \texorpdfstring{$\R^n$}{Rn}}
	
	Consider $M=\R^n$ with the standard Euclidean metric
	\[
	g = \sum_{i=1}^n \dd x^i\otimes\dd x^i.
	\]
	In the standard coordinates $(x^1,\dots,x^n)$ we have $g_{ij}=\delta_{ij}$.
	All partial derivatives $\partial g_{ij}/\partial x^k$ are zero, so all
	Christoffel symbols vanish:
	\[
	\Gamma^k_{ij} = 0 \quad\text{for all } i,j,k.
	\]
	
	The geodesic equation therefore reduces to
	\[
	\frac{\dd^2\gamma^k}{\dd t^2}
	= 0,\qquad k=1,\dots,n.
	\]
	The general solution is
	\[
	\gamma^k(t) = A^k t + B^k,
	\]
	so $\gamma$ is an affine map:
	\[
	\gamma(t) = At + B
	\]
	for some constant vectors $A,B\in\R^n$.  These are precisely straight lines.
	Thus our abstract definition of geodesics recovers the familiar fact that
	geodesics in Euclidean space are straight lines with constant velocity.
	
	\subsection{The round two-sphere \texorpdfstring{$S^2$}{S2}}
	
	Let $M=S^2\subset\R^3$ be the unit sphere with the metric induced by the
	Euclidean inner product.  To compute the geodesic equation, we introduce
	spherical coordinates.
	
	Write a point on $S^2$ as
	\[
	(\cos\theta\cos\varphi,\ \cos\theta\sin\varphi,\ \sin\theta),
	\]
	where $\theta\in(-\tfrac{\pi}{2},\tfrac{\pi}{2})$ is latitude and
	$\varphi\in(0,2\pi)$ is longitude.  In these coordinates, a straightforward
	computation shows that the induced metric is
	\[
	g = \dd\theta^2 + \cos^2\theta\,\dd\varphi^2.
	\]
	(If one uses colatitude instead of latitude, one gets
	$g = \dd\theta^2 + \sin^2\theta\,\dd\varphi^2$; both descriptions are
	equivalent up to a coordinate change.)
	
	For definiteness, let us take
	\[
	g_{\theta\theta}=1,\qquad
	g_{\varphi\varphi}=\sin^2\theta,\qquad
	g_{\theta\varphi}=g_{\varphi\theta}=0.
	\]
	Then the inverse matrix has
	\[
	g^{\theta\theta}=1,\qquad
	g^{\varphi\varphi}=\frac{1}{\sin^2\theta},\qquad
	g^{\theta\varphi}=g^{\varphi\theta}=0.
	\]
	
	Using the formula for Christoffel symbols, we compute the nonzero ones:
	\begin{align*}
		\Gamma^{\theta}_{\varphi\varphi}
		&= \frac12 g^{\theta\theta}
		\left(
		\frac{\partial g_{\varphi\theta}}{\partial x^\varphi}
		+ \frac{\partial g_{\varphi\theta}}{\partial x^\varphi}
		- \frac{\partial g_{\varphi\varphi}}{\partial x^\theta}
		\right)
		= -\sin\theta\cos\theta,\\[0.5em]
		\Gamma^{\varphi}_{\theta\varphi}
		&= \Gamma^{\varphi}_{\varphi\theta}
		= \frac12 g^{\varphi\varphi}
		\left(
		\frac{\partial g_{\varphi\varphi}}{\partial x^\theta}
		+ \frac{\partial g_{\theta\varphi}}{\partial x^\varphi}
		- \frac{\partial g_{\theta\varphi}}{\partial x^\varphi}
		\right)
		= \cot\theta.
	\end{align*}
	All other $\Gamma^k_{ij}$ vanish.
	
	Therefore, a curve $\gamma(t)=(\theta(t),\varphi(t))$ on $S^2$ is a geodesic if
	and only if it satisfies
	\begin{align*}
		\frac{\dd^2\theta}{\dd t^2}
		- \sin\theta\cos\theta
		\left(\frac{\dd\varphi}{\dd t}\right)^2 &= 0,\\[0.5em]
		\frac{\dd^2\varphi}{\dd t^2}
		+ 2\cot\theta\,\frac{\dd\theta}{\dd t}\,\frac{\dd\varphi}{\dd t} &= 0.
	\end{align*}
	
	Solving this system explicitly is somewhat involved, but one can show that its
	solutions are exactly the great circles on $S^2$.  For instance, if we fix a
	unit vector $u\in\R^3$, then the intersection of $S^2$ with the plane through
	the origin orthogonal to $u$ is a great circle.  Parameterizing this curve
	with constant speed yields a solution of the geodesic equation.  Conversely,
	one can prove that any geodesic on the round sphere extends uniquely to a
	great circle.
	
	\subsection{The punctured plane revisited}
	
	We return to the punctured plane
	\[
	M = \R^2 \setminus\{0\}
	\]
	with the induced Euclidean metric.  In local coordinates away from the origin,
	the metric is just the standard Euclidean one, so geodesics are locally
	straight lines.  However, global properties of these geodesics can be more
	complicated.
	
	If $p$ and $q$ lie on the same ray emanating from the origin (say $q=\lambda
	p$ for some $\lambda>0$), then the straight segment from $p$ to $q$ does not
	hit the origin and is a geodesic in $M$ that minimizes length between its
	endpoints.  On the other hand, if $q=-p$ as in
	Example~\ref{ex:punctured-plane-computation}, then any straight segment from
	$p$ to $q$ passes through $0$ and is not contained in $M$.  There are many
	geodesics passing near the origin, but none of them connect $p$ to $q$ in a
	way that realizes the distance $d_g(p,q)$.
	
	This illustrates that the existence of minimizing geodesics between arbitrary
	pairs of points depends on global conditions like completeness.  The punctured
	plane is incomplete, and the failure of geodesics to realize distances between
	some points is one manifestation of this incompleteness.
