	\section{The first variation of energy}
	\label{sec:first-variation}
	
	We now compute the derivative of the energy functional along a variation.
	This is the key step in deriving the geodesic equation.
	
	\subsection{Left and right derivatives}
	
	Because we allow piecewise smooth curves, it is useful to introduce the
	notation for left and right derivatives at the break points.
	
	Let $\gamma\colon[a,b]\to M$ be piecewise $C^\infty$ with partition
	$a=t_0<\dots<t_N=b$ such that $\gamma$ is $C^\infty$ on each open subinterval
	$(t_i,t_{i+1})$.
	
	\begin{definition}\label{dfn:jump}
		For $1\le i\le N-1$, the \emph{left} and \emph{right} derivatives of $\gamma$
		at $t_i$ are defined by
		\[
		\dot\gamma(t_i^-)
		= \lim_{t\to t_i^-} \dot\gamma(t),\qquad
		\dot\gamma(t_i^+)
		= \lim_{t\to t_i^+} \dot\gamma(t),
		\]
		computed in local coordinates.  We define the \emph{jump} at $t_i$ by
		\[
		\Delta_{t_i}\dot\gamma
		= \dot\gamma(t_i^+) - \dot\gamma(t_i^-).
		\]
		We also set
		\[
		\Delta_{t_0}\dot\gamma = \dot\gamma(t_0^+),\qquad
		\Delta_{t_N}\dot\gamma = -\dot\gamma(t_N^-).
		\]
	\end{definition}

	\begin{remark}\label{rmk:4.2}
		If $\gamma $ is $C^\infty $ on the entire interval $[a,b]$, then all jumps vanish except for the jumps at the endpoints $\Delta_{t_0} \dot{\gamma }$ and $\Delta _{t_N}\dot{\Gamma } $.
	\end{remark}
	
	
	\subsection{The first variation formula}
	
	We will repeatedly use the following linear-algebraic lemma.
	
	\begin{lemma}\label{lem:test-fields}
		Let $W\colon[a,b]\to\R^n$ be continuous.  Suppose
		\[
		\int_a^b \ip{V(t)}{W(t)} \dd t = 0
		\]
		for every smooth $V\colon[a,b]\to\R^n$ with $V(a)=V(b)=0$.  Then
		$W(t)\equiv 0$ on $[a,b]$.
	\end{lemma}
	
	\begin{proof}
		Fix $t_0\in(a,b)$ and suppose $W(t_0)\ne 0$.  Let $e_1,\dots,e_n$ be the
		standard basis of $\R^n$, and write $W(t_0) = \sum_{k=1}^n w_k e_k$.  Then
		at least one $w_k$ is nonzero; without loss of generality $w_1\ne 0$.
		
		By continuity, there exists $\delta>0$ such that
		$\ip{e_1}{W(t)}$ has the same sign as $w_1$ for all
		$t\in(t_0-\delta,t_0+\delta)$.  Choose a smooth bump function
		$\varphi\colon[a,b]\to\R$ such that $\varphi$ is supported in
		$(t_0-\delta,t_0+\delta)$ and $\varphi(t)\ge 0$ with $\varphi(t_0)>0$.
		
		Define $V(t) = \varphi(t)e_1$.  Then $V$ is smooth, $V(a)=V(b)=0$, and
		\[
		\int_a^b \ip{V(t)}{W(t)} \dd t
		= \int_a^b \varphi(t)\ip{e_1}{W(t)} \dd t.
		\]
		The integrand is nonnegative and positive near $t_0$, so the integral is
		strictly positive.  This contradicts the assumption that the integral is
		always zero.  Hence $W(t)=0$ for all $t$.
	\end{proof}
	\begin{corollary}
		Let $W : [a,b] \to \mathbb{R} $ be continuous. Suppose that
		\[
			\int _a^b V(t)W(t) \,\mathrm{d} t=0
		\]
		for every smooth $V : [a,b] \to \mathbb{R} $ with $V(a)=V(b)=0$. Then $W(t)\equiv 0$ on $[a,b]$.
	\end{corollary}
	\begin{proof}
		Taking $n=1$ in Lemma \ref{lem:test-fields}, note that $\left\langle V(t),W(t) \right\rangle =V(t)W(t)$.
	\end{proof}
	
	Since manifolds have local coordinates, and thus can be locally identified with $\mathbb{R} ^n$, we can invoke Lemma \ref{lem:test-fields} on manifolds provided we restrict ourselves within a single chart.
	
	Before we state and prove the first variation formula, we introduce some notation.

	\begin{definition}\label{dfn:christoffel-levi-civita-covariant}
		Let $(M,g)$ be a Riemannian $n$-manifold. The \emph{Christoffel symbol} $\Gamma _{ij} ^k$ is defined as
		\[
			\Gamma _{ij} ^k=\frac{1}{2} \sum_{m=1}^ng^{km} \left( \frac{\partial g_{jm} }{\partial x^i} +\frac{\partial g_{im} }{\partial x^j} -\frac{\partial g_{ij} }{\partial x^m}  \right) 
		.\]
		Define the \emph{Levi-Civita connection} $\nabla : \mathfrak{X} (M)\times \mathfrak{X}(M) \to \mathfrak{X} (M)$ on $(M,g)$ as the bilinear map
		\[
			\nabla (\partial_{x^i} ,\partial_{x^j})=\sum_{k=1} ^n \Gamma_{ij} ^k \partial_{x^k}
		.\]
		We generally write $\nabla(X,Y)$ as $\nabla _XY$.
	\end{definition}
	\begin{remark}
		Given $p\in M$, we can evaluate the Christoffel symbols $\Gamma _{ij} ^k(p)$ at the point $p$ by taking a coordinate representation of $g$ in a chart around $p$, and then evaluating the partial dertivatives of $g_{ij} $ at $p$. Moreover, this representation is manifestly independent of the choice of coordinates by invariance of $g$ under choice of coordinates.
	\end{remark}
	For more properties of the Levi-Civita connection, see \cite{lee-riemann} Theorem 5.10. For more generalities on connections and covariant derivatives, see \cite{lee-riemann} chapters 4 and 5. We will only need the coordinate representation of the Levi-Civita connection using the Christoffel symbols for this paper.

	
	\begin{theorem}[First variation of energy]\label{thm:first-variation}
		Let $(M,g)$ be a Riemannian manifold and
		$\gamma\colon[a,b]\to M$ a piecewise $C^\infty$ curve with partition
		$a=t_0<\dots<t_N=b$.  Let $\alpha$ be a variation of $\gamma$ with variation
		vector field $V$.  Then
		\begin{align*}
			\left.\frac{\dd}{\dd u} E_a^b(\bar\alpha(u))\right\vert_{u=0}
			&= -\int_a^b \ipg{V(t)}{\nabla_{\dot\gamma}\dot\gamma(t)} \dd t \\
			&\quad - \sum_{i=0}^N
			\ipg{V(t_i)}{\Delta_{t_i}\dot\gamma}.
		\end{align*}
		Here $\nabla$ is the Levi-Civita connection of $(M,g)$.
	\end{theorem}
	
	\begin{proof}
		We follow the proof sketch given by \cite{spivak}.

		We first express the energy in local coordinates and then differentiate. Choose a coordinate chart $(U,(x^1,\dots,x^n))$ such that
		$\gamma([t_i,t_{i+1}])\subseteq U$.  We may refine the partition if necessary
		to ensure full containment within $U$. Write
		\[
		\gamma(t) = \bigl(\gamma^1(t),\dots,\gamma^n(t)\bigr)
		\]
		and
		\[
		\dot\gamma(t)
		= \sum_{k=1}^n \frac{\dd\gamma^k}{\dd t}(t)
		\frac{\partial}{\partial x^k}\Big\vert_{\gamma(t)}.
		\]
		Let $g_{ij}$ be the components of $g$ in these coordinates.  Then the
		integrand of the energy on $[t_i,t_{i+1}]$ is
		\[
		\frac12\norm{\dot\gamma(t)}_g^2
		= \frac12\sum_{j,\ell=1}^n
		g_{j\ell}(\gamma(t))
		\frac{\dd\gamma^j}{\dd t}(t)
		\frac{\dd\gamma^\ell}{\dd t}(t).
		\]
		Define
		\[
		F(x,y)
		= \frac12 \sum_{j,\ell=1}^n g_{j\ell}(x) y^j y^\ell,
		\qquad x\in U,\ y\in\R^n.
		\]
		Since we are working in coordinates, we can identify $T_{\gamma (t)} U \cong \mathbb{R} ^n$, yielding
		\[
		E_a^b(\gamma)
		= \sum_{i=0}^{N-1}
		\int_{t_i}^{t_{i+1}}
		F\bigl(\gamma(t),\dot\gamma(t)\bigr)\,\dd t.
		\]
		The energy functional now resembles a standard form for which the calculus of variations is applicable.
		
		Now consider a variation $\alpha(u,t)$ with $\alpha(0,t)=\gamma(t)$.  In
		coordinates, write
		\[
		\alpha(u,t) = \bigl(\alpha^1(u,t),\dots,\alpha^n(u,t)\bigr).
		\]
		For each fixed $u$, we have a curve $t\mapsto\alpha(u,t)$, and we denote its
		components by $\alpha ^k(u,t)$.  The velocity is
		\[
		\frac{\partial\alpha}{\partial t}(u,t)
		= \sum_{k=1}^n
		\frac{\partial\alpha^k}{\partial t}(u,t)
		\frac{\partial}{\partial x^k}\Big\vert_{\alpha(u,t)}.
		\]
		
		Then
		\[
		E_a^b(\bar\alpha(u))
		= \sum_{i=0}^{N-1}
		\int_{t_i}^{t_{i+1}}
		F\left(\alpha(u,t),
		\frac{\partial\alpha}{\partial t}(u,t)\right)\,\dd t.
		\]
		Differentiating with respect to $u$ under the integral sign, we get
		\[
		\frac{\dd}{\dd u} E_a^b(\bar\alpha(u))
		= \sum_{i=0}^{N-1}
		\int_{t_i}^{t_{i+1}}
		\left[
		\sum_{k=1}^n
		\frac{\partial F}{\partial x^k}
		\frac{\partial\alpha^k}{\partial u}
		+ \sum_{k=1}^n
		\frac{\partial F}{\partial y^k}
		\frac{\partial}{\partial t}
		\left(\frac{\partial\alpha^k}{\partial u}\right)
		\right]\dd t.
		\]
		Evaluating at $u=0$ and using that $\alpha(0,t)=\gamma(t)$, we find
		\begin{align*}
			\left.\frac{\dd}{\dd u} E_a^b(\bar\alpha(u))\right\vert_{u=0}
			&= \sum_{i=0}^{N-1}
			\int_{t_i}^{t_{i+1}}
			\left[
			\sum_{k=1}^n
			\left.\frac{\partial F}{\partial x^k}\right\vert_{(\gamma,\dot\gamma)}
			\frac{\partial\alpha^k}{\partial u}(0,t)
			\right.\\
			&\qquad\qquad\left.
			+ \sum_{k=1}^n
			\left.\frac{\partial F}{\partial y^k}\right\vert_{(\gamma,\dot\gamma)}
			\frac{\partial}{\partial t}
			\left(\frac{\partial\alpha^k}{\partial u}(0,t)\right)
			\right]\dd t.
		\end{align*}
		We integrate the second term by parts on each interval
		$[t_i,t_{i+1}]$:
		\begin{align*}
			\int_{t_i}^{t_{i+1}}
			\sum_{k=1}^n
			\frac{\partial F}{\partial y^k}
			\frac{\partial}{\partial t}
			\left(\frac{\partial\alpha^k}{\partial u}(0,t)\right)\,\dd t
			&= \left[
			\sum_{k=1}^n
			\frac{\partial F}{\partial y^k}
			\frac{\partial\alpha^k}{\partial u}(0,t)
			\right]_{t_i}^{t_{i+1}} \\
			&\quad - \int_{t_i}^{t_{i+1}}
			\sum_{k=1}^n
			\frac{\partial}{\partial t}
			\left(\frac{\partial F}{\partial y^k}\right)
			\frac{\partial\alpha^k}{\partial u}(0,t)\,\dd t.
		\end{align*}
		
		Summing over $i$ and combining terms, we obtain
		\begin{align*}
			\left.\frac{\dd}{\dd u} E_a^b(\bar\alpha(u))\right\vert_{u=0}
			&= \sum_{i=0}^{N-1}
			\int_{t_i}^{t_{i+1}}
			\sum_{k=1}^n
			\left(
			\frac{\partial F}{\partial x^k}
			- \frac{\partial}{\partial t}
			\left(\frac{\partial F}{\partial y^k}\right)
			\right)
			\frac{\partial\alpha^k}{\partial u}(0,t)\,\dd t \\
			&\quad + \sum_{i=0}^{N-1}
			\left[
			\sum_{k=1}^n
			\frac{\partial F}{\partial y^k}
			\frac{\partial\alpha^k}{\partial u}(0,t)
			\right]_{t_i}^{t_{i+1}}.
		\end{align*}
		
		The last sum telescopes.  Evaluating the boundary terms at the internal points
		$t_1,\dots,t_{N-1}$ yields expressions involving the left and right
		derivatives of $\gamma$ at those points.  One checks that this boundary
		contribution can be written as
		\[
		-\sum_{i=0}^N
		\ipg{V(t_i)}{\Delta_{t_i}\dot\gamma},
		\]
		where $V$ is the variation vector field and $\Delta_{t_i}\dot\gamma$ is the
		jump in the velocity at $t_i$.  We omit some routine algebra; the main point
		is that the terms at $t_i$ involve differences of the form
		\[
		\ipg{V(t_i)}{\dot\gamma(t_i^+)}-
		\ipg{V(t_i)}{\dot\gamma(t_i^-)}
		= \ipg{V(t_i)}{\Delta_{t_i}\dot\gamma}.
		\]
		
		The remaining integral gives the Euler--Lagrange part.  A direct
		computation shows that
		\[
		\frac{\partial F}{\partial x^k}
		- \frac{\partial}{\partial t}
		\left(\frac{\partial F}{\partial y^k}\right)
		= -\ipg{
			\frac{\partial}{\partial x^k}
		}{
			\nabla_{\dot\gamma}\dot\gamma
		},
		\]
		where $\nabla$ is the Levi-Civita connection.  Therefore
		\[
		\sum_{k=1}^n
		\left(
		\frac{\partial F}{\partial x^k}
		- \frac{\partial}{\partial t}
		\left(\frac{\partial F}{\partial y^k}\right)
		\right)
		\frac{\partial\alpha^k}{\partial u}(0,t)
		= -\ipg{V(t)}{\nabla_{\dot\gamma}\dot\gamma(t)}.
		\]
		Substituting this back into the expression for the derivative completes the
		proof.
	\end{proof}
	
