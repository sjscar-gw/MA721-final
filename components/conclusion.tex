	\section*{Conclusion}
	
	In this paper we have developed the variational approach to geodesics on
	Riemannian manifolds.  Starting from the definition of a Riemannian metric, we
	introduced the length and energy functionals on the space of curves, computed
	the first variation of the energy, and derived the geodesic equation as the
	Euler--Lagrange equation for critical points of the energy with fixed
	endpoints.  We saw that geodesics can be characterized intrinsically as curves
	with zero covariant acceleration, and that in local coordinates they satisfy a
	second-order system of ordinary differential equations involving the
	Christoffel symbols.
	
	We also discussed the relationship between energy and length, showing that
	constant-speed geodesics are critical points of the length functional and are
	locally minimizing curves.  Finally, we computed geodesics in several
	important examples, including Euclidean space, the round sphere, and the
	punctured plane, illustrating how both local and global geometric features
	affect the behavior of geodesics.
	
	Many further directions are possible: the study of Jacobi fields and
	conjugate points, Morse theory on loop spaces, comparison theorems relating
	curvature to the behavior of geodesics, and the role of geodesics in physical
	theories such as general relativity.  The variational methods and basic
	geometric ideas developed here provide a foundation for exploring these more
	advanced topics.
