	\section{Variations and the energy functional}
	\label{sec:variations}

	As noted earlier, a geodesic is intuitively understood to be a curve of minimal length, a locally length-minimizing curve. We also noted that this notion does not always agree with the Riemannian distance function, as seen in Example \ref{ex:punctured-plane-computation}. To make precise when these two notions agree, we first provide a classification of geodesics. This classification will be the focus of much of this paper. Much of our presentation for this part follows that of \cite{spivak}.
	
	In order to find critical points of the length or energy functional, we need a
	way to perturb a given curve. For this, we look to the calculus of variations.
	
	\subsection{Variations of curves}
	
	\begin{definition}
		Let $\gamma\colon [a,b]\to M$ be a piecewise $C^\infty$ curve.  A
		\emph{variation} of $\gamma$ is a smooth map
		\[
		\alpha\colon (-\varepsilon,\varepsilon)\times[a,b]\to M
		\]
		for some $\varepsilon>0$ such that:
		\begin{enumerate}[label=(\roman*),leftmargin=*]
			\item $\alpha(0,t) = \gamma(t)$ for all $t\in[a,b]$;
			\item There is a partition
			\[
			a = t_0 < t_1 < \dots < t_N = b
			\]
			such that for each fixed $u$, the curve
			\[
			\bar\alpha(u)\colon [a,b]\to M,\qquad
			\bar\alpha(u)(t) = \alpha(u,t),
			\]
			is $C^\infty$ on each closed subinterval $[t_i,t_{i+1}]$.
		\end{enumerate}
		We say that the variation \emph{fixes endpoints} if $\alpha(u,a)$ and
		$\alpha(u,b)$ are independent of $u$.
	\end{definition}
	
	\begin{definition}
		Let $\alpha$ be a variation of $\gamma$.  The \emph{variation vector field}
		along $\gamma$ is the vector field $V$ along $\gamma$ defined by
		\[
		V(t)
		= \left.\frac{\partial\alpha}{\partial u}\right\vert_{u=0}(t)
		\in T_{\gamma(t)}M.
		\]
		If $\alpha$ fixes endpoints, then $V(a)=V(b)=0$.
	\end{definition}
	
	\begin{remark}
		Conversely, given a sufficiently nice vector field $V$ along $\gamma$, one
		can construct a variation $\alpha$ with variation field $V$.  We will use this
		in the proof of the geodesic equation.
	\end{remark}
	
	\subsection{The energy functional}
	
	Directly working with the length functional
	\[
	L_g(\gamma)
	= \int_a^b \norm{\dot\gamma(t)}_g \dd t
	\]
	is possible but somewhat technical because of the square root in the definition of the norm.  It is more
	convenient to work with the \emph{energy functional}, whose integrand is
	quadratic in the velocity.
	
	\begin{definition}
		Let $\gamma\colon[a,b]\to M$ be piecewise $C^\infty$.  The
		\emph{energy} of $\gamma$ is
		\[
		E_{a}^{b}(\gamma)
		= \frac12 \int_a^b \norm{\dot\gamma(t)}_g^2 \dd t
		= \frac12 \int_a^b
		g_{\gamma(t)}\bigl(\dot\gamma(t),\dot\gamma(t)\bigr)\,\dd t.
		\]
	\end{definition}
	
	If we fix endpoints $p,q\in M$ and an interval $[a,b]$, we can think of
	$E_a^b$ as a functional on the space of piecewise $C^\infty$ curves
	$\gamma\colon[a,b]\to M$ with $\gamma(a)=p$ and $\gamma(b)=q$.  We will say
	that $\gamma$ is a \emph{critical point} of $E_a^b$ if the derivative
	\[
	\left.\frac{\dd}{\dd u} E_a^b(\bar\alpha(u))\right\vert_{u=0}
	\]
	vanishes for every variation $\alpha$ that fixes endpoints.
	
	\begin{proposition}\label{prop:length-energy-ineq}
		Let $\gamma\colon[a,b]\to M$ be piecewise $C^\infty$.  Then
		\[
		L_g(\gamma)^2
		\le 2(b-a)\,E_a^b(\gamma),
		\]
		with equality if and only if $\norm{\dot\gamma(t)}_g$ is constant on $[a,b]$.
	\end{proposition}
	
	\begin{proof}
		By the Cauchy--Schwarz inequality on $L^2$ applied to $\left\| \dot{\gamma } (t) \right\|_g$ and the constant function at 1,
		\[
		\left( \int_a^b \left\| \dot{\gamma }(t) \right\|_g \,\dd t \right)^2
		\leq  (b-a)\int_a^b \left\| \dot{\gamma }(t)  \right\|^2\,\dd t,
		\]
		with equality if and only if $\left| \gamma (t) \right| $ is constant almost everywhere.  The left-hand
		side is $L_g(\gamma)^2$, while the right-hand side is
		\[
		 (b-a)\int_a^b \norm{\dot\gamma(t)}_g^2\,\dd t
		= 2(b-a)\,E_a^b(\gamma).
		\]
	\end{proof}
	
	This shows that among curves with fixed endpoints and fixed parameter
	interval $[a,b]$, those with constant speed are the ones for which energy and
	length are most tightly related.  In particular, any curve that minimizes
	energy among all curves with the same endpoints and parameter interval must
	have constant speed. We will later see that this relation admits a converse in the sense that critical points for $L_g$ are reparameterizations of those for $E_a^b$.
