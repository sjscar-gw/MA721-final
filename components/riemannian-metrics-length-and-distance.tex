	\section{Riemannian metrics, length, and distance}
	\label{sec:prelim}
	
	Throughout, $M$ will denote a smooth ($C^\infty$) manifold of dimension $n$.
	
	\subsection{Riemannian metrics}

	\begin{definition}\label{dfn:riemannian-metric}
		Let $M$ be a smooth manifold. A \emph{Riemannian metric} $g$ on $M$ is a smooth section of $\mathcal{T} ^2(M)=T^*M\otimes T^*M$. More intuitively, $g$ is a smooth assignment to each point
		$p\in M$ of an inner product
		\[
		g_p \colon T_pM \times T_pM \to \R
		\]
		such that the map
		\[
		M \times TM\times TM \to \R,\qquad
		(p,v,w) \mapsto g_p(v,w)
		\]
		is smooth when restricted to $TM\times TM$ over $M$.  We usually write
		$\ipg{v}{w} = g_p(v,w)$ when $v,w\in T_pM$, and
		$\norm{v}_g = \sqrt{\ipg{v}{v}}$. A \emph{Riemannian manifold} is a manifold $M$ together with a Riemannian metric $g$.
	\end{definition}

	All manifolds admit at least one Riemannian metric by \cite{lee} Proposition 13.3, but oftentimes this metric may not be desirable.
	
	\begin{example}
		Let $(U,(x^1,\dots,x^n))$ be a coordinate chart on $M$.  The coordinate vector
		fields $\partial_{x^1},\dots,\partial_{x^n}$ form a basis for
		$T_pM$ at each $p\in U$.  The metric $g$ is then determined by its components
		\[
		g_{ij}(p) = g_p\!\left(
		\frac{\partial}{\partial x^i}\Big\vert_p,
		\frac{\partial}{\partial x^j}\Big\vert_p
		\right),
		\]
		which form a smooth positive-definite symmetric matrix $(g_{ij})$.  We often
		write
		\[
		g = \sum_{i,j=1}^n g_{ij}(x) \dd x^i \otimes \dd x^j.
		\]
	\end{example}

	Unless otherwise specified, all manifolds in this paper are assumed to be Riemannian manifolds with Riemannian metric $g$.
	
	\subsection{Length of curves}
	
	\begin{definition}
		Let $\gamma\colon [a,b]\to M$ a piecewise $C^\infty$ curve on a manifold $M$.  The
		\emph{velocity} of $\gamma$ at $t$ is $\dot\gamma(t) = \dd\gamma/\dd t \in
		T_{\gamma(t)}M$.  The \emph{length} of $\gamma$ with respect to $g$ is
		\[
		L_g(\gamma)
		= \int_a^b \norm{\dot\gamma(t)}_g \dd t
		= \int_a^b \sqrt{
			g_{\gamma(t)}\bigl(\dot\gamma(t),\dot\gamma(t)\bigr)
		}\,\dd t.
		\]
	\end{definition}
	
	It is straightforward to check that the length is invariant under
	orientation-preserving reparametrizations: if $\phi\colon [c,d]\to[a,b]$ is a
	smooth, strictly increasing bijection and $\tilde\gamma = \gamma\circ\phi$,
	then $L_g(\tilde\gamma)=L_g(\gamma)$.
	
	\begin{example}
		In $\R^n$ with the standard metric
		$g = \sum_{i=1}^n \dd x^i\otimes\dd x^i$, the length reduces to the usual
		length formula
		\[
		L_g(\gamma)
		= \int_a^b \sqrt{ \sum_{i=1}^n
			\left(\frac{\dd \gamma^i}{\dd t}(t)\right)^2 }\,\dd t,
		\]
		where $\gamma(t) = (\gamma^1(t),\dots,\gamma^n(t))$.
	\end{example}
	
	\subsection{The Riemannian distance function}
	
	\begin{definition}
		Let $(M,g)$ be a connected Riemannian manifold.  For $p,q\in M$ the
		\emph{Riemannian distance} between $p$ and $q$ is
		\[
		d_g(p,q)
		= \inf\{ L_g(\gamma) \mid \gamma\colon [a,b]\to M
		\text{ piecewise $C^\infty$},\ \gamma(a)=p,\ \gamma(b)=q\}.
		\]
	\end{definition}
	
	We now prove that this defines a metric on $M$.
	
	\begin{proposition}
		The function $d_g\colon M\times M\to\R$ is a metric, i.e.
		\begin{enumerate}[label=(\roman*),leftmargin=*]
			\item $d_g(p,q)\ge 0$ and $d_g(p,q)=0$ if and only if $p=q$,
			\item $d_g(p,q) = d_g(q,p)$, and
			\item $d_g(p,r)\le d_g(p,q)+d_g(q,r)$ for all $p,q,r\in M$.
		\end{enumerate}
	\end{proposition}
	
	\begin{proof}
		(i) By definition, $L_g(\gamma)\ge 0$ for every curve $\gamma$, so
		$d_g(p,q)\ge 0$.  If $p=q$, the constant curve $\gamma(t)\equiv p$ has
		length zero, so $d_g(p,p)=0$. Conversely, suppose $d_g(p,q)=0$.  Let $(U,(x^1,\dots,x^n))$ be a coordinate
		chart containing $p$, and fix the Euclidean norm $\norm{\cdot}$ on $\R^n$.
		Since $g$ is positive definite and smooth, there exist constants
		$0<c<C<\infty$ such that
		\[
		c\norm{v} \le \norm{v}_g \le C\norm{v}
		\]
		for all $v$ in tangent spaces over a small neighborhood of $p$.  If
		$q\ne p$ is sufficiently close to $p$, then there is a smooth curve
		$\gamma$ in $U$ from $p$ to $q$, and its Euclidean length is bounded below by
		some positive number depending on $\norm{x(q)-x(p)}$.  The inequalities above
		imply $L_g(\gamma)\ge c \,L_{\mathrm{Eucl}}(\gamma) >0$, so
		$d_g(p,q)>0$.  Thus if $d_g(p,q)=0$, we must have $p=q$.
		
		(ii) Symmetry is clear: if $\gamma$ is a curve from $p$ to $q$, then the
		reversed curve $\tilde\gamma(t) = \gamma(a+b-t)$ has the same length and goes
		from $q$ to $p$. Taking infima gives $d_g(p,q)=d_g(q,p)$.
		
		(iii) For the triangle inequality, fix $p,q,r\in M$ and $\varepsilon>0$.
		Choose piecewise smooth curves $\gamma_1$ from $p$ to $q$ and $\gamma_2$ from
		$q$ to $r$ such that
		\[
		L_g(\gamma_1) \le d_g(p,q) + \varepsilon,\qquad
		L_g(\gamma_2) \le d_g(q,r) + \varepsilon.
		\]
		Define the concatenated curve
		\[
		\gamma(t)
		= \begin{cases}
			\gamma_1(2t), & t\in[0,\tfrac12],\\
			\gamma_2(2t-1), & t\in[\tfrac12,1].
		\end{cases}
		\]
		Then
		\[
		L_g(\gamma) = L_g(\gamma_1)+L_g(\gamma_2)
		\le d_g(p,q) + d_g(q,r) + 2\varepsilon.
		\]
		Taking the infimum over all such $\gamma$ gives
		$d_g(p,r) \le d_g(p,q) + d_g(q,r) + 2\varepsilon$, and since $\varepsilon>0$
		was arbitrary, we obtain the triangle inequality.
	\end{proof}
	
	It is also true that the metric topology induced by $d_g$ agrees with the
	original manifold topology.  A full proof requires some more work, but the
	idea is that in local coordinates, the inequality
	$c\norm{v}\le\norm{v}_g\le C\norm{v}$ implies that the $d_g$-balls and the
	Euclidean balls define the same notion of ``small neighborhood.''  We will not
	need the precise details later, so see \cite{lee} Theorem 13.29 for a full proof.
