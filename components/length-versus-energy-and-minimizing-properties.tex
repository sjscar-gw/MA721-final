	\section{Length versus energy and minimizing properties}
	\label{sec:length-vs-energy}
	
	We now relate the variational characterization of geodesics (via energy) to
	the more geometric intuition of geodesics as locally length-minimizing curves.
	
	\subsection{Critical points of length vs.\ critical points of energy}
	
	Let $\gamma\colon[a,b]\to M$ be a smooth curve with $\dot\gamma(t)\ne 0$ for
	all $t$.  Define its \emph{arc-length parameter} by
	\[
	s(t) = \int_a^t \norm{\dot\gamma(t)}_g \dd t.
	\]
	Then $s\colon[a,b]\to[0,L_g(\gamma)]$ is smooth and strictly increasing, hence
	a diffeomorphism onto its image.  We can invert it to obtain $t(s)$ and define
	the reparametrized curve
	\[
	\tilde\gamma(s) = \gamma(t(s)).
	\]
	By construction, $\tilde\gamma$ has unit speed:
	\[
	\norm{\dot{\tilde\gamma}(s)}_g
	= 1
	\]
	for all $s$.  In particular, $L_g(\tilde\gamma)=L_g(\gamma)$. We will refer to this as he \emph{arc-length reparameterization} of $\gamma $.
	
	The length functional is less convenient to differentiate because of the
	square root, but for unit-speed curves it is closely related to the energy.
	
	\begin{theorem}\label{thm:length-critical}
		Let $\gamma\colon[a,b]\to M$ be a smooth curve with $\dot\gamma(t)\ne 0$ for
		all $t$, and let $\tilde\gamma$ be its arc-length reparametrization.  Then:
		\begin{enumerate}[label=(\roman*),leftmargin=*]
			\item $\gamma$ is a critical point of the energy functional $E_a^b$ with
			fixed endpoints if and only if its arc-length reparametrization
			$\tilde\gamma$ is a critical point of the length functional.
			\item Any smooth curve $\sigma$ that is critical for the length functional
			(with fixed endpoints) and has nonvanishing velocity can be
			reparametrized to a geodesic.
		\end{enumerate}
	\end{theorem}
	
	\begin{proof}[Sketch of proof]
		A full proof requires computing the first variation of the length functional,
		which is similar to, but slightly more complicated than, the computation for
		the energy.  The idea is as follows.
		
		For a unit-speed curve $\tilde\gamma$, the length and energy functionals
		satisfy
		\[
		L_g(\tilde\gamma)
		= \int_0^{L_g(\gamma)} 1\,\dd s
		= L_g(\gamma),
		\]
		and
		\[
		E_0^{L_g(\gamma)}(\tilde\gamma)
		= \frac12\int_0^{L_g(\gamma)} 1^2\,\dd s
		= \frac12 L_g(\gamma).
		\]
		Thus up to a constant factor, $L$ and $E$ coincide on unit-speed curves.
		The first variation of $L$ along variations that respect unit speed is
		proportional to the first variation of $E$.
		
		More concretely, if $\tilde\alpha$ is a variation of $\tilde\gamma$ through
		unit-speed curves with fixed endpoints, then
		\[
		\left.\frac{\dd}{\dd u} L_g(\tilde\alpha(u))\right\vert_{u=0}
		= \frac{1}{\norm{\dot{\tilde\gamma}}^2_g}
		\left.\frac{\dd}{\dd u}
		E(\tilde\alpha(u))\right\vert_{u=0}.
		\]
		Therefore, the vanishing of the first variation of $E$ is equivalent to the
		vanishing of the first variation of $L$ under such variations.
		
		The second statement follows by applying arc-length reparametrization and then
		using the characterization of geodesics as critical points of $E$ from
		Theorem~\ref{thm:geodesic-iff}.  Details can be filled in by adapting the
		proof of the first variation formula to $L$ instead of $E$.
	\end{proof}
	
	\subsection{Local minimizing property of geodesics}
	
	In general, geodesics are critical points of the length functional, not
	necessarily global minimizers.  Nevertheless, sufficiently short geodesic
	segments do minimize length between their endpoints.
	
	\begin{theorem}[Local minimizing property]
		Let $(M,g)$ be a Riemannian manifold and $p\in M$.  There exists a
		neighborhood $U$ of $p$ such that for any $q\in U$ there is a unique
		geodesic segment $\gamma$ from $p$ to $q$ lying in $U$, and $\gamma$ is the
		unique minimizing curve between $p$ and $q$.
	\end{theorem}
	
	\begin{proof}[Idea of proof]
		The proof uses the exponential map $\exp_p\colon T_pM\to M$, which is defined
		by $\exp_p(v)=\gamma_v(1)$, where $\gamma_v$ is the unique geodesic with
		$\gamma_v(0)=p$ and $\dot\gamma_v(0)=v$.  For sufficiently small $v$, the map
		$\exp_p$ is a diffeomorphism onto a neighborhood $U$ of $p$.
		This induces coordinates in which geodesics through $p$ correspond to straight lines in $T_pM$, and one can use this to show that the radial geodesic from $p$ to
		$q$ is the unique minimizing curve in $U$.
		
		The full proof is beyond the scope of this paper, so we refer the reader to \cite{lee-riemann} chapter 6, and in particular Proposition 6.11.
	\end{proof}
