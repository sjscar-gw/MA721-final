	\section{Geodesics and the geodesic equation}
	\label{sec:geodesic-equation}
	
	We can now define geodesics and characterize them as solutions of a second
	order ODE.
	
	\begin{definition}
		A piecewise $C^\infty$ curve $\gamma\colon[a,b]\to M$ is called a
		\emph{geodesic} if it is a critical point of the energy functional $E_a^b$
		among all variations that fix endpoints.
	\end{definition}
	
	Using the first variation formula, we obtain a more geometric description.
	
	\begin{theorem}\label{thm:geodesic-iff}
		A piecewise $C^\infty$ curve $\gamma\colon[a,b]\to M$ is a critical point of
		$E_a^b$ for fixed endpoints if and only if:
		\begin{enumerate}[label=(\roman*),leftmargin=*]
			\item $\gamma$ is in fact $C^\infty$ on $[a,b]$ (so all jumps
			$\Delta_{t_i}\dot\gamma$ vanish), and
			\item $\gamma$ satisfies the \emph{geodesic equation}
			\[
			\nabla_{\dot\gamma}\dot\gamma(t) = 0
			\quad \text{for all } t\in[a,b].
			\]
		\end{enumerate}
	\end{theorem}
	
	\begin{proof}
		($\implies $) Assume $\gamma$ is a critical point of $E_a^b$.  Let $\alpha$ be any
		variation of $\gamma$ fixing endpoints, with variation vector field $V$.  Then
		by Theorem~\ref{thm:first-variation},
		\[
		0
		= \left.\frac{\dd}{\dd u} E_a^b(\bar\alpha(u))\right\vert_{u=0}
		= -\int_a^b \ipg{V}{\nabla_{\dot\gamma}\dot\gamma} \dd t
		- \sum_{i=0}^N \ipg{V(t_i)}{\Delta_{t_i}\dot\gamma}.
		\]
		
		We first show that the jump terms vanish.  Fix an index $i$ and choose a
		variation $\alpha$ whose variation field $V$ is supported very close to $t_i$,
		with $V(a)=V(b)=0$ and $V(t_j)=0$ for all $j\ne i$ (this can be done by
		building $V$ from a bump function and then integrating to get $\alpha$; see
		remark \ref{rmk:5.3}).  For such a variation, the integral term can be made
		arbitrarily small, but the sum over $i$ reduces to the single term
		$-\ipg{V(t_i)}{\Delta_{t_i}\dot\gamma}$.  Since the whole expression must be
		zero for all such $V$, we conclude that
		\[
		\ipg{V(t_i)}{\Delta_{t_i}\dot\gamma} = 0
		\]
		for all $V(t_i)\in T_{\gamma(t_i)}M$.  This forces $\Delta_{t_i}\dot\gamma=0$.
		
		Thus $\dot\gamma(t)$ has no jumps and is continuous on $[a,b]$.  Since
		$\gamma$ is $C^\infty$ on each subinterval and $\dot\gamma$ is continuous at
		the break points, standard results on ODEs imply that $\gamma$ is in fact
		$C^\infty$ on all of $[a,b]$.
		
		With $\gamma$ now smooth, all $\Delta_{t_i}\dot\gamma=0$, so the first
		variation formula simplifies to
		\[
		\int_a^b \ipg{V}{\nabla_{\dot\gamma}\dot\gamma} \dd t = 0
		\]
		for every smooth vector field $V$ along $\gamma$ with $V(a)=V(b)=0$.  By
		working in local coordinates and applying Lemma~\ref{lem:test-fields}, we see
		that this forces $\nabla_{\dot\gamma}\dot\gamma(t)=0$ for all $t$.
		
		($\impliedby $) Conversely, suppose $\gamma$ is $C^\infty$ and satisfies
		$\nabla_{\dot\gamma}\dot\gamma=0$. Let $\alpha $ be a variation that fixes endpoints. Then all jumps vanish other than the boundary jumps, but $V$ vanishes at the boundary since $\alpha $ fixes endpoints, so the sum component vanishes. The integral
		term in the first variation formula also vanishes since the covariant derivative vanishes. Hence $\dd E_a^b(\bar\alpha(u))/\dd u|_{u=0}=0$ for
		all $\alpha$, so $\gamma$ is a critical point.
	\end{proof}
	
	\begin{remark}\label{rmk:5.3}
		To justify the existence of variations with a prescribed variation field $V$,
		one can proceed as follows.  In local coordinates, define
		\[
		\alpha(u,t)
		= \exp_{\gamma(t)}(uV(t)),
		\]
		where $\exp$ is the exponential map associated to $g$.  For small $u$, this is
		well-defined and yields a smooth map $\alpha$ with $\alpha(0,t)=\gamma(t)$ and
		variation field $V$.  If we want the endpoints fixed, we arrange $V(a)=V(b)=0$.
		We will not develop the full theory of the exponential map here; instead, we
		can work locally and patch together variations using bump functions.
	\end{remark}
	
	In local coordinates, the geodesic equation takes a more explicit form.
	
	\begin{proposition}[Geodesic equation in coordinates]
		A smooth curve $\gamma\colon[a,b]\to M$ with coordinate representation
		$\gamma(t) = (\gamma^1(t),\dots,\gamma^n(t))$ is a geodesic if and only if its
		components satisfy
		\[
		\frac{\dd^2\gamma^k}{\dd t^2}(t)
		+ \sum_{i,j=1}^n
		\Gamma^k_{ij}\bigl(\gamma(t)\bigr)
		\frac{\dd\gamma^i}{\dd t}(t)
		\frac{\dd\gamma^j}{\dd t}(t)
		= 0,
		\qquad k=1,\dots,n.
		\]
	\end{proposition}
	
	\begin{proof}
		Recall definition \ref{dfn:christoffel-levi-civita-covariant} which defines $\nabla $ on basis vectors. It suffices to show that by imposing bilinearity and the product rule, we recover the claimed ODE.
		We have
		\begin{align*}
			\nabla_{\dot{\gamma } } \dot{\gamma } &=\nabla _{\dot{\gamma } } \left( \sum_{i=1}^n \frac{\mathrm{d}\gamma ^i}{\mathrm{d}t} \frac{\partial }{\partial x^i}   \right) \\
							      &=\sum_{i=1}^n \frac{\mathrm{d}\gamma }{\mathrm{d}t} \left( \frac{\mathrm{d}\gamma ^i}{\mathrm{d}t}  \right) \frac{\partial }{\partial x^i} + \frac{\mathrm{d}\gamma ^i}{\mathrm{d}t}\nabla_{\dot{\gamma } } \frac{\partial }{\partial x^i}\\
							      &=\sum_{i=1}^n \frac{\mathrm{d}^2\gamma ^i}{\mathrm{d}t^2}  \frac{\partial }{\partial x^i} +\frac{\mathrm{d}\gamma ^i}{\mathrm{d}t}\nabla_{\dot{\gamma } } \frac{\partial }{\partial x^i} \\
							      &=\sum_{i=1}^n \frac{\mathrm{d}^2\gamma ^i}{\mathrm{d}t^2} \frac{\partial} {\partial x^i} +\frac{\mathrm{d}\gamma ^i}{\mathrm{d}t}  \sum_{j=1}^n \frac{\mathrm{d}\gamma ^j}{\mathrm{d}t}  \nabla_{\partial_{i} } \frac{\partial }{\partial x^j} \\
			&=\sum_{i=1}^n \frac{\mathrm{d}^2\gamma ^i}{\mathrm{d}t^2} \frac{\partial }{\partial x^i} +\sum_{j=1}^n \sum_{k=1}^n \frac{\mathrm{d}\gamma ^i}{\mathrm{d}t} \frac{\mathrm{d}\gamma ^j}{\mathrm{d}t} \Gamma_{ij}^k \frac{\partial }{\partial x^k}\\
			&=\left( \sum_{k=1}^{n} \frac{\mathrm{d}^2\gamma ^k}{\mathrm{d}t^2} \frac{\partial }{\partial x^k}  \right) +\left( \sum_{i=1}^{n} \sum_{j=1}^{n} \sum_{k=1}^{n} \frac{\mathrm{d}\gamma ^i}{\mathrm{d}t}\frac{\mathrm{d}\gamma ^j}{\mathrm{d}t}  \Gamma_{ij}^k \frac{\partial }{\partial x^k}  \right) \\
			&=\sum_{k=1}^n \left( \frac{\mathrm{d}^2\gamma ^k}{\mathrm{d}t^2} +\sum_{i,j=1}^{n} \Gamma_{ij} ^k \frac{\mathrm{d}\gamma ^i}{\mathrm{d}t} \frac{\mathrm{d}\gamma ^j}{\mathrm{d}t}  \right) \frac{\partial }{\partial x^k} 
		.\end{align*}
		
		Here we have noted that $\dot{\gamma } $ acts on $\dot{\gamma } ^i$ by $\dot{\gamma } (\dot{\gamma } ^i)=\ddot{\gamma } ^i$. Thus $\nabla_{\dot\gamma}\dot\gamma=0$ if and only if each coordinate
		component satisfies the claimed ODE.
	\end{proof}
