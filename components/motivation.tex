	\section{Motivation}

	One of the first geometric facts we learn is that, in the Euclidean plane,
	the shortest path between two points is a straight line.  This statement is
	so familiar that it is easy to forget how much structure is hidden in it: we
	are using both the linear structure of $\R^n$ and its standard inner product
	to talk about lengths, angles, and straightness.  As soon as we leave the flat
	world of Euclidean space, the situation becomes less obvious.
	
	For example, on the surface of the Earth, airplanes do not follow straight line
	segments in $\R^3$, but rather arcs of great circles on the sphere.  These
	arcs are locally distance minimizing: near any point on such a curve, if you
	look only at sufficiently short subsegments, they realize the shortest path
	between their endpoints along the surface of the Earth.  From the intrinsic
	point of view of the sphere, they play exactly the role that straight lines
	play in $\R^n$.  This leads to the central question of this paper:
	
	\begin{center}
		\emph{How can we make sense of ``straightest'' or ``shortest'' curves at constant veloctiy on an
			arbitrary smooth manifold?}
	\end{center}
	
	The modern answer begins with the notion of a \emph{Riemannian metric}.
	Informally, a Riemannian metric $g$ on a smooth manifold $M$ gives each
	tangent space $T_pM$ the structure of an inner product space, in a way that
	varies smoothly with the point $p$.  Once such a metric is chosen, we can
	measure the length of tangent vectors, and by integration we obtain lengths of
	curves.  From the lengths of curves we define a distance function $d_g(p,q)$
	by taking an infimum over all curves from $p$ to $q$.
	
	Curves that ``realize'' this distance (or at least make it stationary) are
	called \emph{geodesics}.  Geodesics play the role of straight lines in this
	more general setting.  They are important both for purely geometric reasons
	and for applications: in physics, for instance, geodesics in a Lorentzian
	manifold represent the trajectories of freely-falling particles in general
	relativity.
	
	There are several different but equivalent ways to characterize geodesics:
	
	\begin{itemize}[leftmargin=*]
		\item As locally length-minimizing curves.
		\item As curves with zero covariant acceleration,
		$\nabla_{\dot\gamma}\dot\gamma = 0$.
		\item As critical points of the energy functional
		\[
		E(\gamma)
		= \frac12\int_a^b \norm{\dot\gamma(t)}_g^2 \dd t
		\]
		under variations that fix the endpoints.
	\end{itemize}
	
	The last point is the variational point of view, and it is the main focus of
	this paper.  Instead of guessing the geodesic equation and then justifying it,
	we will start from the length and energy functionals on the space of curves
	and derive the geodesic equation as an Euler--Lagrange equation.  This approach
	fits naturally with classical problems in the calculus of variations and gives
	a conceptually clean derivation of the geodesic equation.
	
	\subsection{A motivating example: the punctured plane}
	
	Before getting into definitions, it is useful to see that the distance function
	behaves in a slightly subtle way even in a very simple example.
	
	\begin{example}[The punctured plane]\label{ex:punctured-plane}
		Let $M = \R^2 \setminus \{0\}$ with the Riemannian metric $g$ induced by
		the standard Euclidean inner product on $\R^2$.  Fix a point
		$p \in M$ and let $q = -p$.  In the full plane $\R^2$, the unique straight
		line segment from $p$ to $q$ has length $2\norm{p}$ and realizes the Euclidean
		distance between $p$ and $q$.
		
		However, in $M$ this straight segment is not allowed, because it passes through
		the origin, which has been removed.  Any piecewise smooth curve
		$\gamma \colon [a,b] \to M$ with $\gamma(a) = p$ and $\gamma(b) = q$ must
		``go around'' the origin.  Intuitively, we still expect the distance between
		$p$ and $q$ to be $2\norm{p}$, but there will be no curve in $M$ that actually
		achieves this length.
	\end{example}
	
	We now make this more precise in one concrete case.
	
	\begin{example}\label{ex:punctured-plane-computation}
		Take $p=(1,0)$ and $q=(-1,0)$, and let $M = \R^2 \setminus \{0\}$ as above.
		For each $\varepsilon > 0$ consider the curve $\gamma_\varepsilon$ that goes
		from $(1,0)$ to $(\varepsilon,0)$ along the $x$-axis, then follows a
		semicircle of radius $\varepsilon$ around the origin to $(-\varepsilon,0)$,
		and then goes from $(-\varepsilon,0)$ to $(-1,0)$ along the $x$-axis.
		
		The first and last segments have total length $2(1-\varepsilon)$, and the
		semicircular arc has length $\pi \varepsilon$.  Therefore
		\[
		L_g(\gamma_\varepsilon)
		= 2(1-\varepsilon) + \pi\varepsilon
		= 2 + (\pi - 2)\varepsilon.
		\]
		As $\varepsilon \to 0^+$ we get $L_g(\gamma_\varepsilon)\to 2$.
		
		On the other hand, every curve in $M$ from $p$ to $q$ must go around the
		origin, and hence has length strictly greater than $2$: if a curve could
		achieve length exactly $2$, it would have to coincide with the straight line
		segment from $(1,0)$ to $(-1,0)$, and that passes through the origin.  Thus
		\[
		d_g(p,q) = 2
		\]
		but there is no curve $\gamma$ in $M$ with $L_g(\gamma) = d_g(p,q)$.  The
		distance is an \emph{infimum} of lengths, not necessarily a \emph{minimum}.
	\end{example}
	
	This example illustrates both the usefulness and the subtlety of the distance
	function.  It also hints at the importance of global assumptions like
	completeness in the study of geodesics.  In complete Riemannian manifolds,
	geodesics are known to realize distances between nearby points, but in
	incomplete manifolds strange things, like the punctured plane, can happen.
	
	\subsection{Goals and outline}
	
	The main goal of this paper is to give a self-contained derivation of the
	geodesic equation on a Riemannian manifold using the calculus of variations,
	and to connect this equation with the intuitive idea of geodesics as shortest
	curves.  Roughly speaking, we will:
	
	\begin{itemize}[leftmargin=*]
		\item Define Riemannian metrics, curve length, and the induced distance
		function $d_g$.
		\item Introduce variations of curves and define the energy functional.
		\item Derive the first variation formula for the energy.
		\item Show that critical points of the energy are precisely smooth curves
		satisfying the geodesic equation $\nabla_{\dot\gamma}\dot\gamma=0$.
		\item Relate geodesics to length-minimizing curves and discuss examples.
	\end{itemize}
	
	The structure of the paper is as follows.  In Section~\ref{sec:prelim} we
	review Riemannian metrics, the length of curves, and the Riemannian distance.
	In Section~\ref{sec:variations} we define variations of curves and the energy
	functional.  Section~\ref{sec:first-variation} is devoted to the first
	variation formula.  In Section~\ref{sec:geodesic-equation} we characterize
	geodesics as curves satisfying a second-order ODE, the geodesic equation, and
	show the equivalence with being critical points of the energy.  In
	Section~\ref{sec:length-vs-energy} we relate energy and length and explain how
	geodesics arise as locally minimizing curves.  Finally, in
	Section~\ref{sec:examples} we compute geodesics explicitly in some important
	examples.
